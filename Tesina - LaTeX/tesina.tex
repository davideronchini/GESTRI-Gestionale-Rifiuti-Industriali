\documentclass[a4paper]{report}

\input{config}

% Document title and author
\title{\huge \textbf{\textcolor{black}{Elettromagnetismo\\per la Trasmissione dell'Informazione}}\\}
\author{Davide Ronchini - 2025}
\date{}

\begin{document}

\thispagestyle{empty} % rimuove intestazione/piede pagina

% decorative thin left band (1.2cm) drawn in overlay so it doesn't change layout
\begin{tikzpicture}[remember picture,overlay]
    % extend a couple millimetres below the page bottom to avoid any tiny gap
    \fill[myGreen] (current page.north west) rectangle ($(current page.south west)+(1.4cm,0)$);
\end{tikzpicture}

\begin{center}
    % Titolo Università
    \vphantom{space} \\[0.5cm]
    {\Large \textbf{\textcolor{myGreen}{GESTRI – Gestionale Rifiuti Industriali}}}\\[0.3cm]
    
    % Corso di Laurea
    {\normalsize Corso di Laurea Triennale in Ingegneria Informatica e dell’Automazione}\\[2cm]
    
    % Logo
    \includegraphics[width=0.35\textwidth]{univpm_logo.png}\\[2cm]
    
    % Titolo progetto
    {\large \textbf{\textcolor{myGreen}{UNIVERSITÀ POLITECNICA DELLE MARCHE}}}\\ [0.3cm]
    % Corso di Laurea
    {\normalsize Corso di Laurea Triennale in Ingegneria Informatica e dell’Automazione}\\[2cm]
    
    % Relatori e Tesina
    \begin{minipage}{0.45\textwidth}
        \begin{flushleft}
        \textbf{\textcolor{myGreen}{RELATORI:}}\\[0.2cm]
        Prof. Ursino Domenico\\
        Prof. Davide Traini
        \end{flushleft}
    \end{minipage}
    \hfill
    \begin{minipage}{0.45\textwidth}
        \begin{flushright}
        \textbf{\textcolor{myGreen}{TESINA DI:}}\\[0.2cm]
        Tarek Naja \\
        Davide Ronchini\\
        Marco Sambughi \\
        Sara Vaccaro
        \end{flushright}
    \end{minipage}\\[5.5cm]
    
    % Anno accademico
    {\normalsize A.A 2024/2025}
\end{center}

% Include the table of contents
\tableofcontents
% Remove page number from table of contents page
\thispagestyle{empty}

\setcounter{page}{0}
\newchapter{Descrizione del Progetto}

\newsection*{Panoramica Generale}
Il progetto si inserisce all’interno del contesto della gestione di un sistema avanzato destinato al monitoraggio e al controllo di un insieme complesso di dati e processi. 
L’obiettivo principale è fornire un’infrastruttura affidabile e centralizzata, capace di gestire in maniera integrata le informazioni provenienti da più fonti e di garantire 
un accesso sicuro e differenziato in base ai ruoli degli utenti.  
Il sistema non si limita alla sola raccolta dei dati, ma prevede una loro elaborazione, archiviazione e presentazione attraverso interfacce chiare e coerenti, 
con particolare attenzione alla scalabilità e alla manutenibilità futura.

\newsection*{Utenti}
Gli utenti del sistema sono suddivisi in due categorie principali.  
La prima è quella degli \textbf{Operatori}, figure incaricate dell’inserimento e dell’aggiornamento delle informazioni di base, che costituiscono il nucleo operativo della piattaforma.  
La seconda è rappresentata dallo \textbf{Staff}, che eredita tutte le funzionalità proprie degli Operatori ma dispone inoltre di strumenti aggiuntivi per il monitoraggio, 
la supervisione e la gestione delle configurazioni globali del sistema.  
Per riassumere in modo uniforme le logiche comuni, è stato introdotto un attore generico denominato \textbf{User}, che rappresenta un’astrazione dei comportamenti condivisi 
tra tutte le tipologie di utenti.

\newsection*{Gestione dei Dati}
Uno dei cardini del progetto è la gestione strutturata dei dati.  
Il sistema deve infatti raccogliere informazioni eterogenee, archiviarle in modo sicuro e consentirne il recupero secondo criteri di rapidità ed efficienza.  
Sono previsti meccanismi di aggiornamento costante e di sincronizzazione, così da garantire la coerenza delle informazioni nel tempo.  
Particolare attenzione è stata posta anche agli aspetti di integrità e consistenza, evitando ridondanze superflue e introducendo controlli atti a prevenire errori nella fase di registrazione.

\newsection*{Architettura del Sistema}
L’architettura del sistema è stata progettata seguendo un approccio modulare, che consente di distinguere chiaramente i diversi livelli funzionali.  
Il livello di acquisizione si occupa di raccogliere i dati dalle fonti esterne, normalizzandoli e predisponendoli all’elaborazione.  
Segue un livello logico-gestionale, in cui le informazioni vengono trattate secondo le regole definite dal dominio applicativo.  
Infine, il livello di presentazione ha il compito di fornire agli utenti una visione chiara e comprensibile dello stato del sistema, adattandosi ai privilegi associati a ciascun ruolo.  
La modularità facilita inoltre l’eventuale estensione futura, consentendo di aggiungere nuove funzionalità senza compromettere la stabilità delle componenti esistenti.

\newsection*{Interfacce e Accesso}
L’accesso al sistema avviene attraverso interfacce pensate per essere intuitive e coerenti, in grado di fornire a ciascun utente esattamente le funzioni necessarie al proprio ruolo.  
Gli Operatori interagiscono principalmente con strumenti di inserimento e aggiornamento, mentre lo Staff dispone di viste aggiuntive che consentono una supervisione complessiva.  
La gestione delle credenziali garantisce la distinzione tra i diversi profili, rafforzando il livello di sicurezza e impedendo utilizzi impropri delle funzionalità disponibili.


\newchapter{Glossario dei Termini}

\begin{longtable}{ >{\raggedright\arraybackslash}p{1.5cm} >{\raggedright\arraybackslash}p{5.5cm} >{\raggedright\arraybackslash}p{2cm} >{\raggedright\arraybackslash}p{2.97cm}}
\toprule
\textbf{TERMINE} & \textbf{DESCRIZIONE} & \textbf{TIPO} & \textbf{SINONIMI} \\
\midrule
\endhead
\bottomrule
\endfoot
\endlastfoot
Utente & Ruolo generico da cui derivano Client e Operator. Ha la capacità di accedere al sistema. & TECNICO & - \\
Client & Soggetto che commissiona il servizio e consulta documenti e stato delle attività. & BUSINESS & - \\
Operator & Figura incaricata di eseguire operazioni pratiche di carico/scarico, compilazione documenti e gestione mezzi. Estende le funzionalità dello Staff. & BUSINESS & - \\
Staff & Personale amministrativo che gestisce l'organizzazione dei turni, le assenze e le attività complessive. & BUSINESS & - \\
Attività & Operazione di carico o scarico di rifiuti, con assegnazione di operatori e mezzi. & BUSINESS & - \\
Mezzo & Veicolo utilizzato per il trasporto dei rifiuti. & BUSINESS & - \\
FIR & Documento obbligatorio che accompagna il trasporto dei rifiuti industriali. & BUSINESS & Formulario di Identificazione Rifiuti \\
Turno & Periodo temporale in cui un operatore è assegnato a un'attività. & BUSINESS & - \\
Gestione Utenti & Area del sistema che si occupa di registrazione, login, e amministrazione dei profili utente. & TECNICO & - \\
Gestione Attività & Area del sistema che gestisce la creazione, l'aggiornamento e il monitoraggio delle operazioni di carico/scarico rifiuti. & TECNICO & - \\
Gestione Documento & Area del sistema che gestisce la creazione, archiviazione e notifica di documenti come il FIR. & TECNICO & - \\
Gestione Mezzo & Area del sistema che tiene traccia dei dati tecnici, assicurativi e di manutenzione dei veicoli. & TECNICO & - \\
MoSCoW & Criterio di prioritizzazione dei requisiti: Must, Should, Could, Won't. & TECNICO & - \\
\end{longtable}

\newchapter{Requisiti}

Nella presente sezione si analizzano in modo approfondito i requisiti del sistema, procedendo alla loro suddivisione secondo le seguenti categorie:
\begin{itemize}
    \item Requisiti funzionali, che definiscono le specifiche funzionalità che il sistema è tenuto a fornire.
    \item Requisiti non funzionali, che individuano i vincoli e le qualità che il sistema deve soddisfare, quali ad esempio l'usabilità, la sicurezza e le prestazioni.
\end{itemize}

\newsection{Requisiti Funzionali}

% Requisiti funzionali (testo inserito automaticamente)
\newsubsection*{Area: Gestione Utente}

\newsubsubsection*{RF1: Registrazione Utente}

Il sistema dovrà permettere la registrazione di nuovi utenti (amministratori, operatori, staff) inserendo i dati necessari.

\newsubsubsection*{RF2: Login Utente}

Il sistema dovrà permettere all’utente di autenticarsi tramite email e password.

\newsubsubsection*{RF3: Assegnazione Ruolo}

Il sistema dovrà permettere l’assegnazione e la modifica del ruolo dell’utente (USER, OPERATOR, STAFF).

\newsubsubsection*{RF4: Modifica Utente}

Il sistema dovrà permettere la modifica delle informazioni anagrafiche e dei dati di contatto di un utente.

\newsubsubsection*{RF5: Elimina Utente}

Il sistema dovrà permettere la cancellazione di un account utente e la gestione coerente dei dati associati.

\newsubsubsection*{RF6: Recupero Credenziali}

Il sistema dovrà permettere il recupero delle credenziali tramite procedura (es. email) per operatori e staff.

\newsubsection*{Area: Gestione Assenza}

\newsubsubsection*{RF7: CRUD Assenza}

Il sistema dovrà permettere allo staff di registrare, approvare e monitorare le assenze del personale.

\newsubsection*{Area: Gestione Attività}

\newsubsubsection*{RF8: CRUD Attività}

Il sistema dovrà permettere la creazione, lettura, aggiornamento e cancellazione delle attività di carico/scarico rifiuti.

\newsubsubsection*{RF9: Assegnazione Operatore}

Il sistema dovrà permettere l'assegnazione di un operatore a ciascuna attività, con controllo delle disponibilità.

\newsubsubsection*{RF10: Assegnazione Mezzo}

Il sistema dovrà permettere l'assegnazione del mezzo più idoneo all'attività in base al tipo di rifiuto e al carico previsto.

\newsubsection*{Area: Gestione Documento}

\newsubsubsection*{RF11: CRUD Documento FIR}

Il sistema dovrà permettere la creazione, lettura, aggiornamento e cancellazione del Formulario di Identificazione Rifiuto (FIR) associato a un'attività assegnata.

\newsubsubsection*{RF11: CRUD Scadenza}

Il sistema dovrà permettere la gestione delle scadenze dei documenti (impostazione, modifica, rimozione) e la loro associazione ai documenti rilevanti.

\newsubsubsection*{RF13: Notifica Scadenza Documento}

Il sistema dovrà inviare notifiche relative alle scadenze dei documenti (es. FIR, consegne, ritiri) agli utenti interessati.

\newsubsubsection*{RF14: Filtra Documento}

Il sistema dovrà permettere il filtraggio e la ricerca dei documenti tramite criteri multipli (tipologia, data, stato, operatore, attività).

\newsubsubsection*{RF15: Notifica Rinnovo Corso Sicurezza}

Il sistema dovrà inviare notifiche per il rinnovo dei corsi di sicurezza agli utenti interessati prima della scadenza.

\newsubsection*{Area: Gestione Mezzo}

\newsubsubsection*{RF16: CRUD Mezzo}

Il sistema dovrà permettere la creazione, lettura, aggiornamento e cancellazione dei mezzi, inclusi dati tecnici, capacità, assicurazione, revisione, manutenzione e stato del mezzo.

\newsubsubsection*{RF17: Filtra Mezzo}

Il sistema dovrà permettere il filtraggio e la ricerca dei mezzi per stato, capacità, disponibilità, scadenze assicurative e altri attributi rilevanti.

\begin{figure*}[ht]
    \centering
    \includegraphics[width=1\textwidth]{RequisitiFunzionali}
\end{figure*}


\vphantom{space}\\ \\ \\ \\ \\ \\ \\ \\ \\ \\
\newsection{Requisiti Non Funzionali}

% Requisiti non funzionali (testo inserito automaticamente)
\newsubsection*{Area: Gestione Tecnologie}

\newsubsubsection*{RNF1: Implementazione in Python 3}

Il sistema dovrà essere implementato utilizzando Python 3.

\newsubsubsection*{RNF2: Utilizzo Database Relazionale}

Il sistema dovrà utilizzare un database relazionale per la gestione persistente dei dati.

\newsubsubsection*{RNF3: Verifica Email}

Il sistema dovrà prevedere la verifica dell'indirizzo email degli utenti durante la registrazione.

\newsubsubsection*{RNF4: Recupero Credenziali Tramite Email}

Il sistema dovrà permettere il recupero delle credenziali tramite email.

\newsubsection*{Area: Gestione UI/UX}

\newsubsubsection*{RNF5: Visualizzazione Turni con Calendario}

Il sistema dovrà offrire una visualizzazione dei turni tramite un calendario integrato.

\newsubsubsection*{RNF6: Interfaccia Responsive}

Il sistema dovrà presentare un'interfaccia responsive, fruibile da dispositivi con diverse risoluzioni.

\newsubsubsection*{RNF7: Notifica Scadenza Compilazione Documento}

Il sistema dovrà inviare notifiche relative alla scadenza della compilazione dei documenti entro 24 ore quando mancano 10 giorni alla scadenza.

\newsubsubsection*{RNF8: Notifica Eliminazione Documento}

Il sistema dovrà inviare notifiche 30 giorni prima dell'eliminazione di un documento al termine del periodo minimo di archiviazione di 3 anni.

\begin{figure*}[ht]
    \centering
    \includegraphics[width=1\textwidth]{RequisitiNonFunzionali}
\end{figure*}


\clearpage
\newsection{Tabella MoSCoW dei Requisiti}

\clearpage
\newsection{Diagrammi dei Casi d'Uso}

\newsubsection{Diagramma degli Attori}
Il diagramma individua gli attori coinvolti nel sistema e visualizza le connessioni di ereditarietà tra di essi.

\begin{figure*}[ht]
    \centering
    \includegraphics[width=1\textwidth]{Attori}
\end{figure*}

\clearpage
\newsubsection*{Gestione Utenti}

Il diagramma evidenzia le relazioni tra attori e casi d’uso per la gestione degli utenti, incluse registrazione, autenticazione, assegnazione di ruoli e modifica dei profili.

\begin{figure*}[ht]
    \centering
    \includegraphics[width=1\textwidth]{GestioneUtente}
\end{figure*}

\clearpage
\begin{table}[t!]
\vspace*{-8cm}
\renewcommand{\arraystretch}{1.9}
\begin{tabular}{|p{3.9cm}|p{9.9cm}|}
\hline
\multicolumn{2}{|c|}{\textbf{Caso d’uso: RegistrazioneUtente}} \\ \hline
\textbf{ID} & 1 \\ \hline
\textbf{Breve descrizione} & L'Utente si registra nel sistema, con un ruolo che dipende dal punto di accesso \\ \hline
\textbf{Attori primari} & Utente, Staff \\ \hline
\textbf{Attori secondari} & Nessuno \\ \hline
\textbf{Precondizioni} & Nessuna \\ \hline
\textbf{Sequenza degli eventi principale} &
\begin{enumerate}[leftmargin=14pt,label=\arabic*.,labelsep=0.5em,topsep=0pt,partopsep=0pt,parsep=0pt,itemsep=0pt]
    \item \texttt{If} l'Utente è un Cliente che accede alla pagina di registrazione pubblica, allora
    \begin{enumerate}[label=\arabic{enumi}.\arabic*.,leftmargin=22pt,labelsep=0.5em,topsep=0pt,partopsep=0pt,parsep=0pt,itemsep=0pt]
        \item Il sistema mostra il modulo di registrazione
        \item Il Cliente inserisce le proprie informazioni (nome, cognome, email, password)
        \item Il sistema valida i dati inseriti
        \item Il sistema crea un nuovo account con il ruolo di Cliente
    \end{enumerate}
\end{enumerate}\\ \hline
\textbf{Postcondizioni} & 
\begin{enumerate}[leftmargin=14pt,label=\arabic*.,labelsep=0.5em,topsep=0pt,partopsep=0pt,parsep=0pt,itemsep=0pt]
\item È stato creato un nuovo account utente con un ruolo specifico (Cliente, Operatore o Staff) 
\end{enumerate} \\ \hline
\textbf{Sequenza degli eventi alternativa} & \begin{enumerate}[leftmargin=14pt,label=\arabic*.,labelsep=0.5em,topsep=0pt,partopsep=0pt,parsep=0pt,itemsep=0pt]
    \item \texttt{If} l'Utente è un membro dello Staff che accede alla pagina di gestione utenti interna, allora
    \begin{enumerate}[label=\arabic{enumi}.\arabic*.,leftmargin=22pt,labelsep=0.5em,topsep=0pt,partopsep=0pt,parsep=0pt,itemsep=0pt]
        \item Il sistema mostra un modulo di creazione utente
        \item Il membro dello Staff inserisce le informazioni del nuovo utente (nome, cognome, email, password) e seleziona il ruolo desiderato (Operatore o Staff)
        \item Il sistema valida i dati inseriti
        \item Il sistema crea un nuovo account con il ruolo specificato (Operatore o Staff)
    \end{enumerate}
\end{enumerate}\\ \hline
\end{tabular}
\end{table}

\clearpage
\begin{table}[t!]
\vspace*{-9cm}
\renewcommand{\arraystretch}{1.9}
\begin{tabular}{|p{3.9cm}|p{9.9cm}|}
\hline
\multicolumn{2}{|c|}{\textbf{Caso d’uso: LoginUtente}} \\ \hline
\textbf{ID} & 2 \\ \hline
\textbf{Breve descrizione} &  Permette all’Utente di accedere al proprio account\\ \hline
\textbf{Attori primari} & Utente \\ \hline
\textbf{Attori secondari} & Nessuno \\ \hline
\textbf{Precondizioni} & \begin{enumerate}[leftmargin=14pt,label=\arabic*.,labelsep=0.5em,topsep=0pt,partopsep=0pt,parsep=0pt,itemsep=0pt]
    \item L’Utente deve essere già registrato nel sistema 
\end{enumerate}\\ \hline
\textbf{Sequenza degli eventi principale} & \begin{enumerate}[leftmargin=14pt,label=\arabic*.,labelsep=0.5em,topsep=0pt,partopsep=0pt,parsep=0pt,itemsep=0pt]
    \item Il caso d’uso inizia quando l’Utente inserirsce le credenziali e seleziona "Accedi" \newline \texttt{Extend} \textit{RecuperaCredenziali}
    \item \texttt{Include} \textit{(TrovaUtente)}
    \item \texttt{While} le credenziali inserite dall’Utente non sono valide
    \begin{enumerate}[label=\arabic{enumi}.\arabic*.,leftmargin=22pt,labelsep=0.5em,topsep=0pt,partopsep=0pt,parsep=0pt,itemsep=0pt]
        \item Il sistema richiede all’Utente di inserire email e password
        \item Il sistema valida le credenziali fornite
    \end{enumerate}
    \item Il sistema autentica l’Utente e lo reindirizza alla pagina principale/area riservata
\end{enumerate}\\ \hline
\textbf{Postcondizioni} & \begin{enumerate}[leftmargin=14pt,label=\arabic*.,labelsep=0.5em,topsep=0pt,partopsep=0pt,parsep=0pt,itemsep=0pt]
    \item L’Utente ha effettuato il login al proprio account e può accedere alle funzionalità autorizzate
    \end{enumerate} \\ \hline
\textbf{Sequenza degli eventi alternativa} & Nessuna \\ \hline
\end{tabular}
\end{table}

\clearpage
\begin{table}[t!]
\vspace*{-6.5cm}
\renewcommand{\arraystretch}{1.9}
\begin{tabular}{|p{3.9cm}|p{9.9cm}|}
\hline
\multicolumn{2}{|c|}{\textbf{Caso d’uso: RecuperaCredenziali}} \\ \hline
\textbf{ID} & 3 \\ \hline
\textbf{Breve descrizione} & L’Utente recupera le credenziali del proprio account in caso di smarrimento o dimenticanza della password \\ \hline
\textbf{Attori primari} & Utente \\ \hline
\textbf{Attori secondari} & Nessuno \\ \hline
\textbf{Precondizioni} & \begin{enumerate}[leftmargin=14pt,label=\arabic*.,labelsep=0.5em,topsep=0pt,partopsep=0pt,parsep=0pt,itemsep=0pt]
    \item L'Utente ha un account registrato nel sistema
    \item L'Utente visualizza la schermata di login
\end{enumerate} \\ \hline
\textbf{Sequenza degli eventi principale} &
\begin{enumerate}[leftmargin=14pt,label=\arabic*.,labelsep=0.5em,topsep=0pt,partopsep=0pt,parsep=0pt,itemsep=0pt]
    \item Il caso d'uso inizia quando l'Utente seleziona l’opzione "Recupera credenziali"
    \item \texttt{While} l'Utente non ha inserito l'email
    \begin{enumerate}[label=\arabic{enumi}.\arabic*.,leftmargin=22pt,labelsep=0.5em,topsep=0pt,partopsep=0pt,parsep=0pt,itemsep=0pt]
        \item Il sistema richiede all’Utente di inserire la propria email
        \item \texttt{Include} \textit{(TrovaUtente)}
        \item Il sistema valida la correttezza e l’esistenza dell’email nel sistema
    \end{enumerate}
    \item Il sistema genera una procedura di recupero (es. invio di un link sicuro o generazione temporanea di una nuova password) e invia le istruzioni all’indirizzo email fornito
    \item L'Utente riceve la mail, segue la procedura indicata e utilizza la nuova password o il link per ripristinare l’accesso al proprio account
\end{enumerate}\\ \hline
\textbf{Postcondizioni} & \begin{enumerate}[leftmargin=14pt,label=\arabic*.,labelsep=0.5em,topsep=0pt,partopsep=0pt,parsep=0pt,itemsep=0pt]
    \item L’Utente ha ricevuto le istruzioni per recuperare l’accesso al proprio account
    \item L’Utente può accedere nuovamente al sistema utilizzando la nuova password o il link fornito
    \end{enumerate} \\ \hline
\textbf{Sequenza degli eventi alternativa} & Nessuna \\ \hline
\end{tabular}
\end{table}

\clearpage
\newsubsection*{Gestione Attività}

Il diagramma evidenzia le relazioni tra attori e casi d’uso per la gestione delle attività (carico/scarico rifiuti), compresa l'assegnazione di operatori e mezzi.

\clearpage
\newsubsection*{Gestione Documento}

Il diagramma evidenzia le relazioni tra attori e casi d’uso relativi alla creazione, gestione e notifica dei documenti (es. FIR, scadenze, rinnovi).

\clearpage
\newsubsection*{Gestione Mezzo}

Il diagramma evidenzia le relazioni tra attori e casi d’uso per la gestione dei mezzi, come anagrafica, manutenzione, disponibilità e scadenze assicurative.

\clearpage
\newsection{Matrice di Mapping}

\newchapter{Analisi}

\newchapter{Progettazione}

\end{document}
