\documentclass[a4paper]{report}

\input{config}

% Document title and author
\title{\huge \textbf{\textcolor{black}{Elettromagnetismo\\per la Trasmissione dell'Informazione}}\\}
\author{Davide Ronchini - 2025}
\date{}

\begin{document}

\thispagestyle{empty} % rimuove intestazione/piede pagina

% decorative thin left band (1.2cm) drawn in overlay so it doesn't change layout
\begin{tikzpicture}[remember picture,overlay]
    % extend a couple millimetres below the page bottom to avoid any tiny gap
    \fill[myGreen] (current page.north west) rectangle ($(current page.south west)+(1.4cm,0)$);
\end{tikzpicture}

\begin{center}
    % Titolo Università
    \vphantom{space} \\[0.5cm]
    {\Large \textbf{\textcolor{myGreen}{GESTRI – Gestionale Rifiuti Industriali}}}\\[0.3cm]
    
    % Corso di Laurea
    {\normalsize Corso di Laurea Triennale in Ingegneria Informatica e dell’Automazione}\\[2cm]
    
    % Logo
    \includegraphics[width=0.35\textwidth]{univpm_logo.png}\\[2cm]
    
    % Titolo progetto
    {\large \textbf{\textcolor{myGreen}{UNIVERSITÀ POLITECNICA DELLE MARCHE}}}\\ [0.3cm]
    % Corso di Laurea
    {\normalsize Corso di Laurea Triennale in Ingegneria Informatica e dell’Automazione}\\[2cm]
    
    % Relatori e Tesina
    \begin{minipage}{0.45\textwidth}
        \begin{flushleft}
        \textbf{\textcolor{myGreen}{RELATORI:}}\\[0.2cm]
        Prof. Ursino Domenico\\
        Prof. Davide Traini
        \end{flushleft}
    \end{minipage}
    \hfill
    \begin{minipage}{0.45\textwidth}
        \begin{flushright}
        \textbf{\textcolor{myGreen}{TESINA DI:}}\\[0.2cm]
        Tarek Naja \\
        Davide Ronchini\\
        Marco Sambughi \\
        Sara Vaccaro
        \end{flushright}
    \end{minipage}\\[5.5cm]
    
    % Anno accademico
    {\normalsize A.A 2024/2025}
\end{center}

% Include the table of contents
\tableofcontents
% Remove page number from table of contents page
\thispagestyle{empty}

\setcounter{page}{0}
\newchapter{Descrizione del Progetto}

\newsection*{Panoramica Generale}
Il progetto si inserisce all’interno del contesto della gestione di un sistema avanzato destinato al monitoraggio e al controllo di un insieme complesso di dati e processi. 
L’obiettivo principale è fornire un’infrastruttura affidabile e centralizzata, capace di gestire in maniera integrata le informazioni provenienti da più fonti e di garantire 
un accesso sicuro e differenziato in base ai ruoli degli utenti.  
Il sistema non si limita alla sola raccolta dei dati, ma prevede una loro elaborazione, archiviazione e presentazione attraverso interfacce chiare e coerenti, 
con particolare attenzione alla scalabilità e alla manutenibilità futura.

\newsection*{Utenti}
Gli utenti del sistema sono suddivisi in due categorie principali.  
La prima è quella degli \textbf{Operatori}, figure incaricate dell’inserimento e dell’aggiornamento delle informazioni di base, che costituiscono il nucleo operativo della piattaforma.  
La seconda è rappresentata dallo \textbf{Staff}, che eredita tutte le funzionalità proprie degli Operatori ma dispone inoltre di strumenti aggiuntivi per il monitoraggio, 
la supervisione e la gestione delle configurazioni globali del sistema.  
Per riassumere in modo uniforme le logiche comuni, è stato introdotto un attore generico denominato \textbf{User}, che rappresenta un’astrazione dei comportamenti condivisi 
tra tutte le tipologie di utenti.

\newsection*{Gestione dei Dati}
Uno dei cardini del progetto è la gestione strutturata dei dati.  
Il sistema deve infatti raccogliere informazioni eterogenee, archiviarle in modo sicuro e consentirne il recupero secondo criteri di rapidità ed efficienza.  
Sono previsti meccanismi di aggiornamento costante e di sincronizzazione, così da garantire la coerenza delle informazioni nel tempo.  
Particolare attenzione è stata posta anche agli aspetti di integrità e consistenza, evitando ridondanze superflue e introducendo controlli atti a prevenire errori nella fase di registrazione.

\newsection*{Architettura del Sistema}
L’architettura del sistema è stata progettata seguendo un approccio modulare, che consente di distinguere chiaramente i diversi livelli funzionali.  
Il livello di acquisizione si occupa di raccogliere i dati dalle fonti esterne, normalizzandoli e predisponendoli all’elaborazione.  
Segue un livello logico-gestionale, in cui le informazioni vengono trattate secondo le regole definite dal dominio applicativo.  
Infine, il livello di presentazione ha il compito di fornire agli utenti una visione chiara e comprensibile dello stato del sistema, adattandosi ai privilegi associati a ciascun ruolo.  
La modularità facilita inoltre l’eventuale estensione futura, consentendo di aggiungere nuove funzionalità senza compromettere la stabilità delle componenti esistenti.

\newsection*{Interfacce e Accesso}
L’accesso al sistema avviene attraverso interfacce pensate per essere intuitive e coerenti, in grado di fornire a ciascun utente esattamente le funzioni necessarie al proprio ruolo.  
Gli Operatori interagiscono principalmente con strumenti di inserimento e aggiornamento, mentre lo Staff dispone di viste aggiuntive che consentono una supervisione complessiva.  
La gestione delle credenziali garantisce la distinzione tra i diversi profili, rafforzando il livello di sicurezza e impedendo utilizzi impropri delle funzionalità disponibili.


\newchapter{Glossario dei Termini}

\begin{longtable}{ >{\raggedright\arraybackslash}p{1.5cm} >{\raggedright\arraybackslash}p{5.5cm} >{\raggedright\arraybackslash}p{2cm} >{\raggedright\arraybackslash}p{2.97cm}}
\toprule
\textbf{TERMINE} & \textbf{DESCRIZIONE} & \textbf{TIPO} & \textbf{SINONIMI} \\
\midrule
\endhead
\bottomrule
\endfoot
\endlastfoot
Utente & Ruolo generico da cui derivano Client e Operator. Ha la capacità di accedere al sistema. & TECNICO & - \\
Client & Soggetto che commissiona il servizio e consulta documenti e stato delle attività. & BUSINESS & - \\
Operator & Figura incaricata di eseguire operazioni pratiche di carico/scarico, compilazione documenti e gestione mezzi. Estende le funzionalità dello Staff. & BUSINESS & - \\
Staff & Personale amministrativo che gestisce l'organizzazione dei turni, le assenze e le attività complessive. & BUSINESS & - \\
Attività & Operazione di carico o scarico di rifiuti, con assegnazione di operatori e mezzi. & BUSINESS & - \\
Mezzo & Veicolo utilizzato per il trasporto dei rifiuti. & BUSINESS & - \\
FIR & Documento obbligatorio che accompagna il trasporto dei rifiuti industriali. & BUSINESS & Formulario di Identificazione Rifiuti \\
Turno & Periodo temporale in cui un operatore è assegnato a un'attività. & BUSINESS & - \\
Gestione Utenti & Area del sistema che si occupa di registrazione, login, e amministrazione dei profili utente. & TECNICO & - \\
Gestione Attività & Area del sistema che gestisce la creazione, l'aggiornamento e il monitoraggio delle operazioni di carico/scarico rifiuti. & TECNICO & - \\
Gestione Documento & Area del sistema che gestisce la creazione, archiviazione e notifica di documenti come il FIR. & TECNICO & - \\
Gestione Mezzo & Area del sistema che tiene traccia dei dati tecnici, assicurativi e di manutenzione dei veicoli. & TECNICO & - \\
MoSCoW & Criterio di prioritizzazione dei requisiti: Must, Should, Could, Won't. & TECNICO & - \\
\end{longtable}

\newchapter{Requisiti}

Nella presente sezione si analizzano in modo approfondito i requisiti del sistema, procedendo alla loro suddivisione secondo le seguenti categorie:
\begin{itemize}
    \item Requisiti funzionali, che definiscono le specifiche funzionalità che il sistema è tenuto a fornire.
    \item Requisiti non funzionali, che individuano i vincoli e le qualità che il sistema deve soddisfare, quali ad esempio l'usabilità, la sicurezza e le prestazioni.
\end{itemize}

\newsection{Requisiti Funzionali}

% Requisiti funzionali (testo inserito automaticamente)
\newsubsection*{Area: Gestione Utente}

\newsubsubsection*{RF1: Registrazione Utente}

Il sistema dovrà permettere la registrazione di nuovi utenti (amministratori, operatori, staff) inserendo i dati necessari.

\newsubsubsection*{RF2: Login Utente}

Il sistema dovrà permettere all’utente di autenticarsi tramite email e password.

\newsubsubsection*{RF3: Recupero Credenziali}

Il sistema dovrà permettere il recupero delle credenziali tramite procedura (es. email) per operatori e staff.

\newsubsubsection*{RF4: Visualizza Utente}

Il sistema dovrà permettere all'Utente di visualizzare i propri dati anagrafici e le informazioni correlate (ruolo, turni, assenze, attività); lo Staff dovrà poter visualizzare i dati di altri utenti per scopi di gestione e supervisione.

\newsubsubsection*{RF5: Modifica Ruolo}

Il sistema dovrà permettere allo Staff di assegnare e modificare il ruolo dell'utente (USER, OPERATOR, STAFF).

\newsubsubsection*{RF6: Modifica Utente}

Il sistema dovrà permettere la modifica delle informazioni anagrafiche e dei dati di contatto di un utente.

\newsubsubsection*{RF7: Elimina Utente}

Il sistema dovrà permettere la cancellazione di un account utente e la gestione coerente dei dati associati.

\newsubsection*{Area: Gestione Assenza}

\newsubsubsection*{RF8: CRUD Assenza}

Il sistema dovrà permettere allo staff di registrare, approvare e monitorare le assenze del personale.

\newsubsection*{Area: Gestione Attività}

\newsubsubsection*{RF9: CRUD Attività}

Il sistema dovrà permettere la creazione, lettura, aggiornamento e cancellazione delle attività di carico/scarico rifiuti.

\newsubsubsection*{RF10: Assegnazione Operatore}

Il sistema dovrà permettere l'assegnazione di un operatore a ciascuna attività, con controllo delle disponibilità.

\newsubsubsection*{RF11: Assegnazione Mezzo}

Il sistema dovrà permettere l'assegnazione del mezzo più idoneo all'attività in base al tipo di rifiuto e al carico previsto.

\newsubsection*{Area: Gestione Documento}

\newsubsubsection*{RF12: CRUD Documento FIR}

Il sistema dovrà permettere la creazione, lettura, aggiornamento e cancellazione del documento (es. Formulario di Identificazione Rifiuto - FIR) associato a un'attività assegnata.

\newsubsubsection*{RF13: Filtra Documento}

Il sistema dovrà permettere il filtraggio e la ricerca dei documenti tramite criteri multipli (tipologia, data, stato, operatore, attività).

\newsubsubsection*{RF14: Notifica Rinnovo Corso Sicurezza}

Il sistema dovrà inviare notifiche per il rinnovo dei corsi di sicurezza agli utenti interessati prima della scadenza.

\newsubsubsection*{RF15: Notifica Scadenza Documento}

Il sistema dovrà inviare notifiche relative alle scadenze dei documenti (es. FIR, consegne, ritiri) agli utenti interessati.

\newsubsection*{Area: Gestione Mezzo}

\newsubsubsection*{RF16: CRUD Mezzo}

Il sistema dovrà permettere la creazione, lettura, aggiornamento e cancellazione dei mezzi, inclusi dati tecnici, capacità, assicurazione, revisione, manutenzione e stato del mezzo.

\newsubsubsection*{RF17: Filtra Mezzo}

Il sistema dovrà permettere il filtraggio e la ricerca dei mezzi per stato, capacità, disponibilità, scadenze assicurative e altri attributi rilevanti.

\begin{figure*}[ht]
    \centering
    \includegraphics[width=1\textwidth]{RequisitiFunzionali}
\end{figure*}


\vphantom{space}\\[6cm]
\newsection{Requisiti Non Funzionali}

% Requisiti non funzionali (testo inserito automaticamente)
\newsubsection*{Area: Gestione Tecnologie}

\newsubsubsection*{RNF1: Implementazione in Python 3}

Il sistema dovrà essere implementato utilizzando Python 3.

\newsubsubsection*{RNF2: Utilizzo Database Relazionale}

Il sistema dovrà utilizzare un database relazionale per la gestione persistente dei dati.

\newsubsubsection*{RNF3: Verifica Email}

Il sistema dovrà prevedere la verifica dell'indirizzo email degli utenti durante la registrazione.

\newsubsubsection*{RNF4: Recupero Credenziali Tramite Email}

Il sistema dovrà permettere il recupero delle credenziali tramite email.

\newsubsection*{Area: Gestione UI/UX}

\newsubsubsection*{RNF5: Visualizzazione Turni con Calendario}

Il sistema dovrà offrire una visualizzazione dei turni tramite un calendario integrato.

\newsubsubsection*{RNF6: Interfaccia Responsive}

Il sistema dovrà presentare un'interfaccia responsive, fruibile da dispositivi con diverse risoluzioni.

\newsubsubsection*{RNF7: Notifica Scadenza Compilazione Documento}

Il sistema dovrà inviare notifiche relative alla scadenza della compilazione dei documenti entro 24 ore quando mancano 10 giorni alla scadenza.

\newsubsubsection*{RNF8: Notifica Eliminazione Documento}

Il sistema dovrà inviare notifiche 30 giorni prima dell'eliminazione di un documento al termine del periodo minimo di archiviazione di 3 anni.

\begin{figure*}[ht]
    \centering
    \includegraphics[width=1\textwidth]{RequisitiNonFunzionali}
\end{figure*}


\clearpage
\newsection{Tabella MoSCoW dei Requisiti}

Di seguito la matrice MoSCoW che classifica i requisiti del sistema secondo le priorità: M (Must-have), S (Should-have), C (Could-have), W (Won't-have).

\begin{longtable}{|p{4.5cm}|p{6.25cm}|p{2cm}|}
\hline
	\textbf{Area} & \textbf{Requisito} & \textbf{Priorità MoSCoW} \\
\hline
\endhead

Gestione Utenti & RF1: Registrazione Utente & Must \\
\hline
Gestione Utenti & RF2: Login Utente & Must \\
\hline
Gestione Utenti & RF3: Recupero Credenziali & Should \\
\hline
Gestione Utenti & RF4: Visualizza Utente & Must \\
\hline
Gestione Utenti & RF5: Modifica Ruolo & Should \\
\hline
Gestione Utenti & RF6: Modifica Utente & Should \\
\hline
Gestione Utenti & RF7: Elimina Utente & Must \\
\hline
Gestione Assenza & RF8: CRUD Assenza & Should \\
\hline
Gestione Attività & RF9: CRUD Attività - carico/scarico & Must \\
\hline
Gestione Attività & RF10: Assegnazione Operatore & Must \\
\hline
Gestione Attività & RF11: Assegnazione Mezzo & Must \\
\hline
Gestione Documento & RF12: CRUD Documento FIR & Must \\
\hline
Gestione Documento & RF13: Filtra Documento & Should \\
\hline
Gestione Documento & RF14: Notifica Rinnovo Corso Sicurezza & Could \\
\hline
Gestione Documento & RF15: Notifica Scadenza Documento & Should \\
\hline
Gestione Mezzo & RF16: CRUD Mezzo & Must \\
\hline
Gestione Mezzo & RF17: Filtra Mezzo & Should \\
\hline
Gestione Tecnologie & RNF1: Implementazione in Python 3 & Must \\
\hline
Gestione Tecnologie & RNF2: Utilizzo Database Relazionale & Must \\
\hline
Gestione Tecnologie & RNF3: Verifica Email & Should \\
\hline
Gestione Tecnologie & RNF4: Recupero Credenziali tramite Email & Should \\
\hline
Gestione UI/UX & RNF5: Visualizzazione Turni con Calendario & Could \\
\hline
Gestione UI/UX & RNF6: Interfaccia Responsive & Should \\
\hline
Gestione UI/UX & RNF7: Notifica Scadenza Compilazione Documento & Should \\
\hline
Gestione UI/UX & RNF8: Notifica Eliminazione Documento & Won't \\
\hline
\end{longtable}

\clearpage
\newsection{Diagrammi dei Casi d'Uso}

\newsubsection*{Diagramma degli Attori}
Il diagramma individua gli attori coinvolti nel sistema e visualizza le connessioni di ereditarietà tra di essi.

\begin{figure*}[ht]
    \centering
    \includegraphics[width=1\textwidth]{Attori}
\end{figure*}

\clearpage
\newsubsection*{Gestione Utenti}

Il diagramma evidenzia le relazioni tra attori e casi d’uso per la gestione degli utenti, incluse registrazione, autenticazione, assegnazione di ruoli e modifica dei profili.

\begin{figure*}[ht]
    \centering
    \includegraphics[width=1\textwidth]{GestioneUtente}
\end{figure*}

\clearpage
\begin{table}[H]
\vspace*{-0cm}
\renewcommand{\arraystretch}{1.9}
\begin{tabular}{|p{3.9cm}|p{9.9cm}|}
\hline
\multicolumn{2}{|c|}{\textbf{Caso d’uso: RegistrazioneUtente}} \\ \hline
\textbf{ID} & 1 \\ \hline
\textbf{Breve descrizione} & L'Utente si registra nel sistema, con un ruolo che dipende dal punto di accesso \\ \hline
\textbf{Attori primari} & Utente, Staff \\ \hline
\textbf{Attori secondari} & Nessuno \\ \hline
\textbf{Precondizioni} & Nessuna \\ \hline
\textbf{Sequenza degli eventi principale} &
\begin{enumerate}[leftmargin=14pt,label=\arabic*.,labelsep=0.5em,topsep=0pt,partopsep=0pt,parsep=0pt,itemsep=0pt]
    \item \texttt{If} l'Utente è un Cliente che accede alla pagina di registrazione pubblica, allora
    \begin{enumerate}[label=\arabic{enumi}.\arabic*.,leftmargin=22pt,labelsep=0.5em,topsep=0pt,partopsep=0pt,parsep=0pt,itemsep=0pt]
        \item Il sistema mostra il modulo di registrazione
        \item Il Cliente inserisce le proprie informazioni (nome, cognome, email, password)
        \item Il sistema valida i dati inseriti
        \item Il sistema crea un nuovo account con il ruolo di Cliente
    \end{enumerate}
\end{enumerate}\\ \hline
\textbf{Postcondizioni} & 
\begin{enumerate}[leftmargin=14pt,label=\arabic*.,labelsep=0.5em,topsep=0pt,partopsep=0pt,parsep=0pt,itemsep=0pt]
\item È stato creato un nuovo account utente con un ruolo specifico (Cliente, Operatore o Staff) 
\end{enumerate} \\ \hline
\textbf{Sequenza degli eventi alternativa} & \begin{enumerate}[leftmargin=14pt,label=\arabic*.,labelsep=0.5em,topsep=0pt,partopsep=0pt,parsep=0pt,itemsep=0pt]
    \item \texttt{If} l'Utente è un membro dello Staff che accede alla pagina di gestione utenti interna, allora
    \begin{enumerate}[label=\arabic{enumi}.\arabic*.,leftmargin=22pt,labelsep=0.5em,topsep=0pt,partopsep=0pt,parsep=0pt,itemsep=0pt]
        \item Il sistema mostra un modulo di creazione utente
        \item Il membro dello Staff inserisce le informazioni del nuovo utente (nome, cognome, email, password) e seleziona il ruolo desiderato (Operatore o Staff)
        \item Il sistema valida i dati inseriti
        \item Il sistema crea un nuovo account con il ruolo specificato (Operatore o Staff)
    \end{enumerate}
\end{enumerate}\\ \hline
\end{tabular}
\end{table}

\clearpage
\begin{table}[H]
\vspace*{-0cm}
\renewcommand{\arraystretch}{1.9}
\begin{tabular}{|p{3.9cm}|p{9.9cm}|}
\hline
\multicolumn{2}{|c|}{\textbf{Caso d’uso: LoginUtente}} \\ \hline
\textbf{ID} & 2 \\ \hline
\textbf{Breve descrizione} &  Permette all’Utente di accedere al proprio account\\ \hline
\textbf{Attori primari} & Utente \\ \hline
\textbf{Attori secondari} & Nessuno \\ \hline
\textbf{Precondizioni} & \begin{enumerate}[leftmargin=14pt,label=\arabic*.,labelsep=0.5em,topsep=0pt,partopsep=0pt,parsep=0pt,itemsep=0pt]
    \item L’Utente deve essere già registrato nel sistema 
\end{enumerate}\\ \hline
\textbf{Sequenza degli eventi principale} & \begin{enumerate}[leftmargin=14pt,label=\arabic*.,labelsep=0.5em,topsep=0pt,partopsep=0pt,parsep=0pt,itemsep=0pt]
    \item Il caso d’uso inizia quando l’Utente inserirsce le credenziali e seleziona "Accedi" \newline \texttt{Extend} \textit{RecuperaCredenziali}
    \item \texttt{While} le credenziali inserite dall’Utente non sono valide
    \begin{enumerate}[label=\arabic{enumi}.\arabic*.,leftmargin=22pt,labelsep=0.5em,topsep=0pt,partopsep=0pt,parsep=0pt,itemsep=0pt]
        \item Il sistema richiede all’Utente di inserire email e password
        \item Il sistema valida le credenziali fornite
    \end{enumerate}
    \item Il sistema autentica l’Utente e lo reindirizza alla pagina principale/area riservata
\end{enumerate}\\ \hline
\textbf{Postcondizioni} & \begin{enumerate}[leftmargin=14pt,label=\arabic*.,labelsep=0.5em,topsep=0pt,partopsep=0pt,parsep=0pt,itemsep=0pt]
    \item L’Utente ha effettuato il login al proprio account e può accedere alle funzionalità autorizzate
    \end{enumerate} \\ \hline
\textbf{Sequenza degli eventi alternativa} & Nessuna \\ \hline
\end{tabular}
\end{table}

\clearpage
\begin{table}[H]
\vspace*{-0cm}
\renewcommand{\arraystretch}{1.9}
\begin{tabular}{|p{3.9cm}|p{9.9cm}|}
\hline
\multicolumn{2}{|c|}{\textbf{Caso d’uso: RecuperaCredenziali}} \\ \hline
\textbf{ID} & 3 \\ \hline
\textbf{Breve descrizione} & L’Utente recupera le credenziali del proprio account in caso di smarrimento o dimenticanza della password \\ \hline
\textbf{Attori primari} & Utente \\ \hline
\textbf{Attori secondari} & Nessuno \\ \hline
\textbf{Precondizioni} & \begin{enumerate}[leftmargin=14pt,label=\arabic*.,labelsep=0.5em,topsep=0pt,partopsep=0pt,parsep=0pt,itemsep=0pt]
    \item L'Utente ha un account registrato nel sistema
    \item L'Utente visualizza la schermata di login
\end{enumerate} \\ \hline
\textbf{Sequenza degli eventi principale} &
\begin{enumerate}[leftmargin=14pt,label=\arabic*.,labelsep=0.5em,topsep=0pt,partopsep=0pt,parsep=0pt,itemsep=0pt]
    \item Il caso d'uso inizia quando l'Utente seleziona l’opzione "Recupera credenziali"
    \item \texttt{While} l'Utente non ha inserito l'email
    \begin{enumerate}[label=\arabic{enumi}.\arabic*.,leftmargin=22pt,labelsep=0.5em,topsep=0pt,partopsep=0pt,parsep=0pt,itemsep=0pt]
        \item Il sistema richiede all’Utente di inserire la propria email
        \item Il sistema valida la correttezza e l’esistenza dell’email nel sistema
    \end{enumerate}
    \item Il sistema genera una procedura di recupero (es. invio di un link sicuro o generazione temporanea di una nuova password) e invia le istruzioni all’indirizzo email fornito
    \item L'Utente riceve la mail, segue la procedura indicata e utilizza la nuova password o il link per ripristinare l’accesso al proprio account
\end{enumerate}\\ \hline
\textbf{Postcondizioni} & \begin{enumerate}[leftmargin=14pt,label=\arabic*.,labelsep=0.5em,topsep=0pt,partopsep=0pt,parsep=0pt,itemsep=0pt]
    \item L’Utente ha ricevuto le istruzioni per recuperare l’accesso al proprio account
    \item L’Utente può accedere nuovamente al sistema utilizzando la nuova password o il link fornito
    \end{enumerate} \\ \hline
\textbf{Sequenza degli eventi alternativa} & Nessuna \\ \hline
\end{tabular}
\end{table}


\clearpage
\begin{table}[H]
\vspace*{-0cm}
\renewcommand{\arraystretch}{1.9}
\begin{tabular}{|p{3.9cm}|p{9.9cm}|}
\hline
\multicolumn{2}{|c|}{\textbf{Caso d’uso: TrovaUtente}} \\ \hline
\textbf{ID} & 4 \\ \hline
\textbf{Breve descrizione} & Il sistema individua l’ Utente in base ai criteri di ricerca specificati dallo Staff e lo mostra allo Staff \\ \hline
\textbf{Attori primari} & Staff \\ \hline
\textbf{Attori secondari} & Nessuno \\ \hline
\textbf{Precondizioni} & \begin{enumerate}[leftmargin=14pt,label=\arabic*.,labelsep=0.5em,topsep=0pt,partopsep=0pt,parsep=0pt,itemsep=0pt]
    \item Lo Staff è stato autenticato dal sistema
\end{enumerate} \\ \hline
\textbf{Sequenza degli eventi principale} &
\begin{enumerate}[leftmargin=14pt,label=\arabic*.,labelsep=0.5em,topsep=0pt,partopsep=0pt,parsep=0pt,itemsep=0pt]
    \item Il caso d’uso inizia quando lo Staff seleziona “Trova Utente”
    \item Lo Staff inserisce i criteri di ricerca
    \item Il sistema ricerca l’Utente che soddisfa i criteri desiderati dallo Staff
    \item \texttt{If} Il sistema trova uno o più utenti
    \begin{enumerate}[label=\arabic{enumi}.\arabic{enumii}.,leftmargin=22pt,labelsep=0.5em,topsep=0pt,partopsep=0pt,parsep=0pt,itemsep=0pt]
    \item \texttt{For Each} Utente trovato
    \begin{enumerate}[label=\arabic{enumi}.\arabic{enumii}.\arabic{enumiii}.,leftmargin=22pt,labelsep=0.5em,topsep=0pt,partopsep=0pt,parsep=0pt,itemsep=0pt]
        \item Il sistema mostra le informazioni base dell’Utente
    \end{enumerate}
    \end{enumerate}
    \item \texttt{Else} Il sistema comunica allo staff che non sono stati trovati Utenti che soddisfano i criteri specificati
\end{enumerate}\\ \hline
\textbf{Postcondizioni} & Nessuna \\ \hline
\textbf{Sequenza degli eventi alternativa} & Nessuna \\ \hline
\end{tabular}
\end{table}

\clearpage
\begin{table}[H]
\vspace*{-0cm}
\renewcommand{\arraystretch}{1.9}
\begin{tabular}{|p{3.9cm}|p{9.9cm}|}
\hline
\multicolumn{2}{|c|}{\textbf{Caso d’uso: VisualizzaUtente}} \\ \hline
\textbf{ID} & 5 \\ \hline
\textbf{Breve descrizione} & L’Utente visualizza i propri dati anagrafici e le informazioni \\ \hline
\textbf{Attori primari} & Utente \\ \hline
\textbf{Attori secondari} & Nessuno \\ \hline
\textbf{Precondizioni} & \begin{enumerate}[leftmargin=14pt,label=\arabic*.,labelsep=0.5em,topsep=0pt,partopsep=0pt,parsep=0pt,itemsep=0pt]
    \item L’Utente è stato autenticato dal sistema
\end{enumerate} \\ \hline
\textbf{Sequenza degli eventi principale} &
\begin{enumerate}[leftmargin=14pt,label=\arabic*.,labelsep=0.5em,topsep=0pt,partopsep=0pt,parsep=0pt,itemsep=0pt]
    \item \texttt{Include} \textit{(TrovaUtente)} 
    \item Il sistema mostra i dati anagrafici  e le informazioni relative all’Utente
\end{enumerate}\\ \hline
\textbf{Postcondizioni} & \begin{enumerate}[leftmargin=14pt,label=\arabic*.,labelsep=0.5em,topsep=0pt,partopsep=0pt,parsep=0pt,itemsep=0pt]
    \item All’Utente sono mostrati i propri dati
    \end{enumerate} \\ \hline
\textbf{Sequenza degli eventi alternativa} & \begin{enumerate}[leftmargin=14pt,label=\arabic*.,labelsep=0.5em,topsep=0pt,partopsep=0pt,parsep=0pt,itemsep=0pt] 
    \item \texttt{If} Lo Staff richiede la visualizzazione
    \begin{enumerate}[label=\arabic{enumi}.\arabic*.,leftmargin=22pt,labelsep=0.5em,topsep=0pt,partopsep=0pt,parsep=0pt,itemsep=0pt]
        \item \texttt{Include} \textit{(TrovaUtente)} 
        \item Il sistema mostra l’elenco di tutti gli Operatori registrati
        \item Lo Staff seleziona un Operatore dall’elenco
        \item Il sistema mostra i dati anagrafici completi e le informazioni aggiuntive (ruolo, turni, assenze, attività corrente collegata)
    \end{enumerate}
\end{enumerate}\\ \hline
\end{tabular}
\end{table}

\clearpage
\begin{table}[H]
\vspace*{-0cm}
\renewcommand{\arraystretch}{1.9}
\begin{tabular}{|p{3.9cm}|p{9.9cm}|}
\hline
\multicolumn{2}{|c|}{\textbf{Caso d’uso: ModificaRuolo}} \\ \hline
\textbf{ID} & 6 \\ \hline
\textbf{Breve descrizione} & Lo Staff modifica il ruolo all’Operatore e/o allo Staff \\ \hline
\textbf{Attori primari} & Staff \\ \hline
\textbf{Attori secondari} & Nessuno \\ \hline
\textbf{Precondizioni} & \begin{enumerate}[leftmargin=14pt,label=\arabic*.,labelsep=0.5em,topsep=0pt,partopsep=0pt,parsep=0pt,itemsep=0pt]
    \item Lo Staff è autenticato
    \item Lo Staff visualizza la sezione interna per la gestione degli utenti
\end{enumerate} \\ \hline
\textbf{Sequenza degli eventi principale} &
\begin{enumerate}[leftmargin=14pt,label=\arabic*.,labelsep=0.5em,topsep=0pt,partopsep=0pt,parsep=0pt,itemsep=0pt]
    \item Il caso d’uso inizia quando lo Staff seleziona l’Operatore da modificare
    \item \texttt{Include} \textit{(VisualizzaUtente)}
    \item Lo Staff può modificare il ruolo dell’Operatore
\end{enumerate}\\ \hline
\textbf{Postcondizioni} & \begin{enumerate}[leftmargin=14pt,label=\arabic*.,labelsep=0.5em,topsep=0pt,partopsep=0pt,parsep=0pt,itemsep=0pt]
    \item Il ruolo dell’Operatore è stato modificato
    \end{enumerate} \\ \hline
\textbf{Sequenza degli eventi alternativa} & Nessuna\\ \hline
\end{tabular}
\end{table}

\clearpage
\begin{table}[H]
\vspace*{-0cm}
\renewcommand{\arraystretch}{1.9}
\begin{tabular}{|p{3.9cm}|p{9.9cm}|}
\hline
\multicolumn{2}{|c|}{\textbf{Caso d’uso: ModificaUtente}} \\ \hline
\textbf{ID} & 7 \\ \hline
\textbf{Breve descrizione} &  L’Utente modifica i propri dati anagrafici \\ \hline
\textbf{Attori primari} & Utente \\ \hline
\textbf{Attori secondari} & Nessuno \\ \hline
\textbf{Precondizioni} & \begin{enumerate}[leftmargin=14pt,label=\arabic*.,labelsep=0.5em,topsep=0pt,partopsep=0pt,parsep=0pt,itemsep=0pt]
    \item L’Utente è stato autenticato dal sistema
\end{enumerate} \\ \hline
\textbf{Sequenza degli eventi principale} &
\begin{enumerate}[leftmargin=14pt,label=\arabic*.,labelsep=0.5em,topsep=0pt,partopsep=0pt,parsep=0pt,itemsep=0pt]
    \item \texttt{Include} \textit{(VisualizzaUtente)} 
    \item l’Utente modifica i dati desiderati
    \item \texttt{If} I dati inseriti sono validi
    \begin{enumerate}[label=\arabic{enumi}.\arabic*.,leftmargin=22pt,labelsep=0.5em,topsep=0pt,partopsep=0pt,parsep=0pt,itemsep=0pt]
        \item Il sistema aggiorna i dati anagrafici con le modifiche
    \end{enumerate}
    \item \texttt{Else}
    \begin{enumerate}[label=\arabic{enumi}.\arabic*.,leftmargin=22pt,labelsep=0.5em,topsep=0pt,partopsep=0pt,parsep=0pt,itemsep=0pt]
        \item Il sistema mostra un messaggio di errore e richiede una nuova modifica
    \end{enumerate}
\end{enumerate}\\ \hline
\textbf{Postcondizioni} & \begin{enumerate}[leftmargin=14pt,label=\arabic*.,labelsep=0.5em,topsep=0pt,partopsep=0pt,parsep=0pt,itemsep=0pt]
    \item I dati anagrafici dell’Utente vengono aggiornati nel sistema
    \end{enumerate} \\ \hline
\textbf{Sequenza degli eventi alternativa} & \begin{enumerate}[leftmargin=14pt,label=\arabic*.,labelsep=0.5em,topsep=0pt,partopsep=0pt,parsep=0pt,itemsep=0pt] 
    \item \texttt{If} Lo Staff richiede la modifica
    \begin{enumerate}[label=\arabic{enumi}.\arabic*.,leftmargin=22pt,labelsep=0.5em,topsep=0pt,partopsep=0pt,parsep=0pt,itemsep=0pt]
        \item \texttt{Include} \textit{(VisualizzaUtente)} 
        \item Il sistema mostra l’elenco di tutti gli Operatori registrati
        \item Lo Staff seleziona l’Operatore da modificare dall’elenco
        \item Lo Staff modifica i dati desiderati
        \item \texttt{If} I dati inseriti sono validi
        \begin{enumerate}[label=\arabic{enumi}.\arabic*.,leftmargin=22pt,labelsep=0.5em,topsep=0pt,partopsep=0pt,parsep=0pt,itemsep=0pt]
            \item Il sistema aggiorna i dati anagrafici con le modifiche
        \end{enumerate}
        \item \texttt{Else}
        \begin{enumerate}[label=\arabic{enumi}.\arabic*.,leftmargin=22pt,labelsep=0.5em,topsep=0pt,partopsep=0pt,parsep=0pt,itemsep=0pt]
            \item Il sistema mostra un messaggio di errore e richiede una nuova modifica
        \end{enumerate}
    \end{enumerate}
\end{enumerate}\\ \hline
\end{tabular}
\end{table}

\clearpage
\begin{table}[H]
\vspace*{-0cm}
\renewcommand{\arraystretch}{1.9}
\begin{tabular}{|p{3.9cm}|p{9.9cm}|}
\hline
\multicolumn{2}{|c|}{\textbf{Caso d’uso: EliminaUtente}} \\ \hline
\textbf{ID} & 8 \\ \hline
\textbf{Breve descrizione} &  L’Utente elimina il proprio account \\ \hline
\textbf{Attori primari} & Utente \\ \hline
\textbf{Attori secondari} & Nessuno \\ \hline
\textbf{Precondizioni} & \begin{enumerate}[leftmargin=14pt,label=\arabic*.,labelsep=0.5em,topsep=0pt,partopsep=0pt,parsep=0pt,itemsep=0pt]
    \item L’Utente è stato autenticato dal sistema
\end{enumerate} \\ \hline
\textbf{Sequenza degli eventi principale} &
\begin{enumerate}[leftmargin=14pt,label=\arabic*.,labelsep=0.5em,topsep=0pt,partopsep=0pt,parsep=0pt,itemsep=0pt]
    \item \texttt{Include} \textit{(VisualizzaUtente)} 
    \item \texttt{If} L’Utente conferma l’operazione
    \begin{enumerate}[label=\arabic{enumi}.\arabic*.,leftmargin=22pt,labelsep=0.5em,topsep=0pt,partopsep=0pt,parsep=0pt,itemsep=0pt]
        \item Il sistema elimina l’Utente e mostra un messaggio di conferma
    \end{enumerate}
    \item \texttt{Else}
    \begin{enumerate}[label=\arabic{enumi}.\arabic*.,leftmargin=22pt,labelsep=0.5em,topsep=0pt,partopsep=0pt,parsep=0pt,itemsep=0pt]
        \item Il sistema annulla l’operazione e non apporta modifiche
    \end{enumerate}
\end{enumerate}\\ \hline
\textbf{Postcondizioni} & \begin{enumerate}[leftmargin=14pt,label=\arabic*.,labelsep=0.5em,topsep=0pt,partopsep=0pt,parsep=0pt,itemsep=0pt]
    \item \texttt{If} confermato
    \begin{enumerate}[label=\arabic{enumi}.\arabic*.,leftmargin=22pt,labelsep=0.5em,topsep=0pt,partopsep=0pt,parsep=0pt,itemsep=0pt]
        \item L’Utente selezionato viene rimosso dal sistema
    \end{enumerate}
\end{enumerate} \\ \hline
\textbf{Sequenza degli eventi alternativa} & \begin{enumerate}[leftmargin=14pt,label=\arabic*.,labelsep=0.5em,topsep=0pt,partopsep=0pt,parsep=0pt,itemsep=0pt] 
    \item \texttt{If} Lo Staff richiede l'eliminazione
    \begin{enumerate}[label=\arabic{enumi}.\arabic*.,leftmargin=22pt,labelsep=0.5em,topsep=0pt,partopsep=0pt,parsep=0pt,itemsep=0pt]
        \item \texttt{Include} \textit{(VisualizzaUtente)} 
        \item Il sistema mostra l’elenco di tutti gli Operatori registrati
        \item Lo Staff seleziona l’Operatore da eliminare dall’elenco
        \item \texttt{If} Lo Staff conferma l’operazione
        \begin{enumerate}[label=\arabic{enumi}.\arabic*.,leftmargin=22pt,labelsep=0.5em,topsep=0pt,partopsep=0pt,parsep=0pt,itemsep=0pt]
            \item Il sistema elimina l’Utente e mostra un messaggio di conferma
        \end{enumerate}
        \item \texttt{Else}
        \begin{enumerate}[label=\arabic{enumi}.\arabic*.,leftmargin=22pt,labelsep=0.5em,topsep=0pt,partopsep=0pt,parsep=0pt,itemsep=0pt]
            \item Il sistema annulla l’operazione e non apporta modifiche
        \end{enumerate}
    \end{enumerate}
\end{enumerate}\\ \hline
\end{tabular}
\end{table}



\clearpage
\newsubsection*{Gestione Assenza}

La sottosezione descrive i casi d'uso relativi alla gestione delle assenze del personale (ferie, malattia, permessi). In questa area sono documentate le operazioni principali: creazione, visualizzazione, modifica ed eliminazione delle assenze, con l'indicazione degli attori coinvolti (principalmente lo Staff) e delle precondizioni necessarie.

\begin{figure*}[ht]
    \centering
    \includegraphics[width=1\textwidth]{GestioneAssenza}
\end{figure*}

\clearpage
\renewcommand{\arraystretch}{1.9}
\begin{table}[H]
\vspace*{-0cm}
\begin{tabular}{|p{3.9cm}|p{9.9cm}|}
\hline
\multicolumn{2}{|c|}{\textbf{Caso d’uso: CreaAssenza}} \\ \hline
\textbf{ID} & 9 \\ \hline
\textbf{Breve descrizione} & Lo Staff crea una nuova assenza (ferie, malattia, permesso) per un Operatore. \\ \hline
\textbf{Attori primari} & Staff \\ \hline
\textbf{Attori secondari} & Nessuno \\ \hline
\textbf{Precondizioni} & \begin{enumerate}[leftmargin=14pt,label=\arabic*.,labelsep=0.5em,topsep=0pt,partopsep=0pt,parsep=0pt,itemsep=0pt]
    \item Lo Staff è autenticato.
    \item L’Operatore per cui viene registrata l’assenza esiste
\end{enumerate} \\ \hline
\textbf{Sequenza degli eventi principale} & \begin{enumerate}[leftmargin=14pt,label=\arabic*.,labelsep=0.5em,topsep=0pt,partopsep=0pt,parsep=0pt,itemsep=0pt]
    \item Il caso d’uso inizia quando lo Staff seleziona “Nuova assenza”
    \item Il sistema richiede i dati dell’assenza (tipo, data inizio, data fine)
    \item Lo Staff inserisce i dati
    \item \texttt{If} I dati sono validi
    \begin{enumerate}[label=\arabic{enumi}.\arabic*.,leftmargin=22pt,labelsep=0.5em,topsep=0pt,partopsep=0pt,parsep=0pt,itemsep=0pt]
        \item Il sistema registra l’assenza
        \item Il sistema mostra un messaggio di conferma
    \end{enumerate}
    \item \texttt{Else}
    \begin{enumerate}[label=\arabic{enumi}.\arabic*.,leftmargin=22pt,labelsep=0.5em,topsep=0pt,partopsep=0pt,parsep=0pt,itemsep=0pt]
        \item Il sistema mostra un errore e richiede la correzione
    \end{enumerate}
\end{enumerate} \\ \hline
\textbf{Postcondizioni} & \begin{enumerate}[label=\arabic*.,leftmargin=14pt,labelsep=0.5em,topsep=0pt,partopsep=0pt,parsep=0pt,itemsep=0pt]
        \item L’assenza è registrata nel sistema.
\end{enumerate} \\ \hline
\textbf{Sequenza alternativa} & \begin{enumerate}[leftmargin=14pt,label=\arabic*.,labelsep=0.5em,topsep=0pt,partopsep=0pt,parsep=0pt,itemsep=0pt]
    \item \texttt{If} Lo Staff non conferma l’operazione
    \begin{enumerate}[label=\arabic{enumi}.\arabic*.,leftmargin=22pt,labelsep=0.5em,topsep=0pt,partopsep=0pt,parsep=0pt,itemsep=0pt]
        \item Il sistema annulla l’operazione e non apporta modifiche.
    \end{enumerate}
\end{enumerate} \\ \hline
\end{tabular}
\end{table}

\clearpage
\begin{table}[H]
\vspace*{-0cm}
\renewcommand{\arraystretch}{1.9}
\begin{tabular}{|p{3.9cm}|p{9.9cm}|}
\hline
\multicolumn{2}{|c|}{\textbf{Caso d’uso: VisualizzaAssenza}} \\ \hline
\textbf{ID} & 10 \\ \hline
\textbf{Breve descrizione} & L’Operatore visualizza le assenze registrate. \\ \hline
\textbf{Attori primari} & Operatore \\ \hline
\textbf{Attori secondari} & Nessuno \\ \hline
\textbf{Precondizioni} & \begin{enumerate}[leftmargin=14pt,label=\arabic*.,labelsep=0.5em,topsep=0pt,partopsep=0pt,parsep=0pt,itemsep=0pt]
    \item L’Operatore è autenticato.
    \item Esistono assenze registrate.
\end{enumerate} \\ \hline
\textbf{Sequenza degli eventi principale} & \begin{enumerate}[leftmargin=14pt,label=\arabic*.,labelsep=0.5em,topsep=0pt,partopsep=0pt,parsep=0pt,itemsep=0pt]
    \item Il caso d’uso inizia quando l’Operatore accede alla sezione “Assenze”.
    \item Il sistema mostra tutte le assenze registrate.
    \item L’operatore seleziona un’assenza specifica.
    \item Il sistema mostra i dettagli dell’assenza selezionata
\end{enumerate} \\ \hline
\textbf{Postcondizioni} & \begin{enumerate}[label=\arabic*.,leftmargin=14pt,labelsep=0.5em,topsep=0pt,partopsep=0pt,parsep=0pt,itemsep=0pt]
        \item L’Operatore visualizza le informazioni sulle assenze.
\end{enumerate} \\ \hline
\textbf{Sequenza degli eventi alternativa} & Nessuna. \\ \hline
\end{tabular}
\end{table}

\clearpage
\begin{table}[H]
\vspace*{-0cm}
\renewcommand{\arraystretch}{1.9}
\begin{tabular}{|p{3.9cm}|p{9.9cm}|}
\hline
\multicolumn{2}{|c|}{\textbf{Caso d’uso: ModificaAssenza}} \\ \hline
\textbf{ID} & 11 \\ \hline
\textbf{Breve descrizione} & Lo Staff modifica i dati di un’assenza esistente. \\ \hline
\textbf{Attori primari} & Staff \\ \hline
\textbf{Attori secondari} & Nessuno \\ \hline
\textbf{Precondizioni} & \begin{enumerate}[leftmargin=14pt,label=\arabic*.,labelsep=0.5em,topsep=0pt,partopsep=0pt,parsep=0pt,itemsep=0pt]
    \item Lo Staff è autenticato.
    \item L’assenza esiste nel sistema.
\end{enumerate} \\ \hline
\textbf{Sequenza degli eventi principale} & \begin{enumerate}[leftmargin=14pt,label=\arabic*.,labelsep=0.5em,topsep=0pt,partopsep=0pt,parsep=0pt,itemsep=0pt]
    \item \texttt{Include} \textit{(VisualizzaAssenza)}.
    \item Lo Staff aggiorna i dati.
    \item \texttt{If} I dati modificati sono validi
    \begin{enumerate}[label=\arabic{enumi}.\arabic*.,leftmargin=22pt,labelsep=0.5em,topsep=0pt,partopsep=0pt,parsep=0pt,itemsep=0pt]
        \item Il sistema aggiorna l’assenza.
    \end{enumerate}
    \item \texttt{Else}
    \begin{enumerate}[label=\arabic{enumi}.\arabic*.,leftmargin=22pt,labelsep=0.5em,topsep=0pt,partopsep=0pt,parsep=0pt,itemsep=0pt]
        \item Il sistema mostra un errore e non applica le modifiche.
    \end{enumerate}
\end{enumerate} \\ \hline
\textbf{Postcondizioni} & \begin{enumerate}[label=\arabic*.,leftmargin=14pt,labelsep=0.5em,topsep=0pt,partopsep=0pt,parsep=0pt,itemsep=0pt]
        \item L’assenza è aggiornata nel sistema.
\end{enumerate} \\ \hline
\textbf{Sequenza alternativa} & \begin{enumerate}[leftmargin=14pt,label=\arabic*.,labelsep=0.5em,topsep=0pt,partopsep=0pt,parsep=0pt,itemsep=0pt]
    \item \texttt{If} Lo Staff non conferma l’operazione
    \begin{enumerate}[label=\arabic{enumi}.\arabic*.,leftmargin=22pt,labelsep=0.5em,topsep=0pt,partopsep=0pt,parsep=0pt,itemsep=0pt]
        \item Il sistema annulla l’operazione e non apporta modifiche.
    \end{enumerate}
\end{enumerate} \\ \hline
\end{tabular}
\end{table}

\clearpage
\begin{table}[H]
\vspace*{-0cm}
\renewcommand{\arraystretch}{1.9}
\begin{tabular}{|p{3.9cm}|p{9.9cm}|}
\hline
\multicolumn{2}{|c|}{\textbf{Caso d’uso: EliminaAssenza}} \\ \hline
\textbf{ID} & 12 \\ \hline
\textbf{Breve descrizione} & Lo Staff elimina un’assenza registrata. \\ \hline
\textbf{Attori primari} & Staff \\ \hline
\textbf{Attori secondari} & Nessuno \\ \hline
\textbf{Precondizioni} & \begin{enumerate}[leftmargin=14pt,label=\arabic*.,labelsep=0.5em,topsep=0pt,partopsep=0pt,parsep=0pt,itemsep=0pt]
    \item Lo Staff è autenticato.
    \item L’assenza esiste nel sistema.
\end{enumerate} \\ \hline
\textbf{Sequenza degli eventi principale} & \begin{enumerate}[leftmargin=14pt,label=\arabic*.,labelsep=0.5em,topsep=0pt,partopsep=0pt,parsep=0pt,itemsep=0pt]
    \item \texttt{Include} \textit{(VisualizzaAssenza)}.
    \item Lo Staff seleziona l’assenza da eliminare.
\end{enumerate} \\ \hline
\textbf{Postcondizioni} & \begin{enumerate}[label=\arabic*.,leftmargin=14pt,labelsep=0.5em,topsep=0pt,partopsep=0pt,parsep=0pt,itemsep=0pt]
        \item L’assenza è eliminata.
\end{enumerate} \\ \hline
\textbf{Sequenza alternativa} & \begin{enumerate}[leftmargin=14pt,label=\arabic*.,labelsep=0.5em,topsep=0pt,partopsep=0pt,parsep=0pt,itemsep=0pt]
    \item \texttt{If} Lo Staff non conferma l’operazione
    \begin{enumerate}[label=\arabic{enumi}.\arabic*.,leftmargin=22pt,labelsep=0.5em,topsep=0pt,partopsep=0pt,parsep=0pt,itemsep=0pt]
        \item Il sistema annulla l’operazione e non apporta modifiche.
    \end{enumerate}
\end{enumerate} \\ \hline
\end{tabular}
\end{table}

\clearpage
\newsubsection*{Gestione Mezzo}

Il diagramma evidenzia le relazioni tra attori e casi d’uso per la gestione dei mezzi, come anagrafica, manutenzione, disponibilità e scadenze assicurative.

\begin{figure*}[ht]
    \centering
    \includegraphics[width=1\textwidth]{GestioneMezzo}
\end{figure*}

\clearpage
\begin{table}[H]
\vspace*{-0cm}
\renewcommand{\arraystretch}{1.9}
\begin{tabular}{|p{3.9cm}|p{9.9cm}|}
\hline
\multicolumn{2}{|c|}{\textbf{Caso d’uso: CreaMezzo}} \\ \hline
	\textbf{ID} & 13 \\ \hline
	\textbf{Breve descrizione} & Lo Staff crea un nuovo mezzo nel sistema. \\ \hline
	\textbf{Attori primari} & Staff \\ \hline
	\textbf{Attori secondari} & Nessuno \\ \hline
	\textbf{Precondizioni} & \begin{enumerate}[label=\arabic*.,leftmargin=14pt,labelsep=0.5em,topsep=0pt,partopsep=0pt,parsep=0pt,itemsep=0pt]
        \item Lo Staff è stato autenticato dal sistema.
    \end{enumerate} \\ \hline
	\textbf{Sequenza degli eventi principale} &
\begin{enumerate}[leftmargin=14pt,label=\arabic*.,labelsep=0.5em,topsep=0pt,partopsep=0pt,parsep=0pt,itemsep=0pt]
    \item Il caso d’uso inizia quando lo Staff seleziona “Nuovo Mezzo”.
    \item Il sistema richiede l’inserimento dei dati (targa, modello, anno, categoria, stato).
    \item Lo Staff inserisce i dati.
    \item \texttt{If} i dati inseriti sono validi e non duplicati
    \begin{enumerate}[label=\arabic{enumi}.\arabic*.,leftmargin=22pt,labelsep=0.5em,topsep=0pt,partopsep=0pt,parsep=0pt,itemsep=0pt]
        \item Il sistema registra il nuovo mezzo.
        \item Il sistema mostra un messaggio di conferma.
    \end{enumerate}
    \item \texttt{Else}
    \begin{enumerate}[label=\arabic{enumi}.\arabic*.,leftmargin=22pt,labelsep=0.5em,topsep=0pt,partopsep=0pt,parsep=0pt,itemsep=0pt]
        \item Il sistema mostra un messaggio di errore e richiede la correzione dei dati.
    \end{enumerate}
\end{enumerate}\\ \hline
    	\textbf{Postcondizioni} & \begin{enumerate}[label=\arabic*.,leftmargin=14pt,labelsep=0.5em,topsep=0pt,partopsep=0pt,parsep=0pt,itemsep=0pt]
        \item Il nuovo mezzo è registrato nel sistema.
        \end{enumerate} \\ \hline
    	\textbf{Sequenza degli eventi alternativa} &
\begin{enumerate}[leftmargin=14pt,label=\arabic*.,labelsep=0.5em,topsep=0pt,partopsep=0pt,parsep=0pt,itemsep=0pt]
    \item \texttt{If} lo Staff non conferma l’operazione
    \begin{enumerate}[label=\arabic{enumi}.\arabic*.,leftmargin=22pt,labelsep=0.5em,topsep=0pt,partopsep=0pt,parsep=0pt,itemsep=0pt]
        \item Il sistema annulla l’operazione e non apporta modifiche.
    \end{enumerate}
\end{enumerate} \\ \hline
\end{tabular}
\end{table}

\clearpage
\begin{table}[H]
\vspace*{-0cm}
\renewcommand{\arraystretch}{1.9}
\begin{tabular}{|p{3.9cm}|p{9.9cm}|}
\hline
\multicolumn{2}{|c|}{\textbf{Caso d’uso: TrovaMezzo}} \\ \hline
	\textbf{ID} & 14 \\ \hline
	\textbf{Breve descrizione} & Lo Staff ricerca un mezzo tramite criteri (targa, modello, stato, ecc.). \\ \hline
	\textbf{Attori primari} & Staff \\ \hline
	\textbf{Attori secondari} & Nessuno \\ \hline
	\textbf{Precondizioni} & \begin{enumerate}[label=\arabic*.,leftmargin=14pt,labelsep=0.5em,topsep=0pt,partopsep=0pt,parsep=0pt,itemsep=0pt]
        \item Lo Staff è stato autenticato dal sistema.
    \end{enumerate} \\ \hline
	\textbf{Sequenza degli eventi principale} & 
\begin{enumerate}[leftmargin=14pt,label=\arabic*.,labelsep=0.5em,topsep=0pt,partopsep=0pt,parsep=0pt,itemsep=0pt]
    \item Il caso d’uso inizia quando lo Staff inserisce un criterio di ricerca nella barra dedicata. \newline 
    \texttt{Extend} \textit{FiltraMezzo}
    \item \texttt{While} il criterio di ricerca non è vuoto
    \begin{enumerate}[label=\arabic{enumi}.\arabic*.,leftmargin=22pt,labelsep=0.5em,topsep=0pt,partopsep=0pt,parsep=0pt,itemsep=0pt]
        \item Il sistema trova i mezzi corrispondenti.
        \item \texttt{If} viene trovato almeno un mezzo
        \begin{enumerate}[label=\arabic{enumi}.\arabic{enumii}.\arabic*.,leftmargin=22pt,labelsep=0.5em,topsep=0pt,partopsep=0pt,parsep=0pt,itemsep=0pt]
            \item Il sistema mostra l’elenco dei mezzi corrispondenti.
        \end{enumerate}
        \item \texttt{Else}
        \begin{enumerate}[label=\arabic{enumi}.\arabic{enumii}.\arabic*.,leftmargin=22pt,labelsep=0.5em,topsep=0pt,partopsep=0pt,parsep=0pt,itemsep=0pt]
            \item Il sistema mostra il messaggio “Nessun mezzo trovato”.
        \end{enumerate}
    \end{enumerate}
\end{enumerate}\\ \hline
    	\textbf{Postcondizioni} & \begin{enumerate}[label=\arabic*.,leftmargin=14pt,labelsep=0.5em,topsep=0pt,partopsep=0pt,parsep=0pt,itemsep=0pt]
        \item Lo Staff visualizza l’elenco dei mezzi corrispondenti ai criteri.
        \end{enumerate} \\ \hline
    	\textbf{Sequenza degli eventi alternativa} & Nessuna. \\ \hline
\end{tabular}
\end{table}


\clearpage
\begin{table}[H]
\vspace*{-0cm}
\renewcommand{\arraystretch}{1.9}
\begin{tabular}{|p{3.9cm}|p{9.9cm}|}
\hline
\multicolumn{2}{|c|}{\textbf{Caso d’uso: FiltraMezzo}} \\ \hline
	\textbf{ID} & 15 \\ \hline
	\textbf{Breve descrizione} & Lo Staff filtra i mezzi in base a criteri avanzati (targa, modello, stato, ecc.). \\ \hline
	\textbf{Attori primari} & Staff \\ \hline
	\textbf{Attori secondari} & Nessuno \\ \hline
	\textbf{Precondizioni} & \begin{enumerate}[label=\arabic*.,leftmargin=14pt,labelsep=0.5em,topsep=0pt,partopsep=0pt,parsep=0pt,itemsep=0pt]
        \item Lo Staff è stato autenticato dal sistema.
    \end{enumerate} \\ \hline
	\textbf{Sequenza degli eventi principale} & 
\begin{enumerate}[leftmargin=14pt,label=\arabic*.,labelsep=0.5em,topsep=0pt,partopsep=0pt,parsep=0pt,itemsep=0pt]
    \item \texttt{Include} \textit{(TrovaMezzo)}.\item Lo Staff applica uno o più filtri.
    \item \texttt{If} esistono mezzi che rispettano i filtri
    \begin{enumerate}[label=\arabic{enumi}.\arabic*.,leftmargin=22pt,labelsep=0.5em,topsep=0pt,partopsep=0pt,parsep=0pt,itemsep=0pt]
        \item Il sistema mostra l’elenco dei mezzi filtrati.
    \end{enumerate}
    \item \texttt{Else}
    \begin{enumerate}[label=\arabic{enumi}.\arabic*.,leftmargin=22pt,labelsep=0.5em,topsep=0pt,partopsep=0pt,parsep=0pt,itemsep=0pt]
        \item Il sistema mostra “Nessun mezzo corrisponde ai criteri di filtro”.
    \end{enumerate}
\end{enumerate}\\ \hline
    	\textbf{Postcondizioni} & \begin{enumerate}[label=\arabic*.,leftmargin=14pt,labelsep=0.5em,topsep=0pt,partopsep=0pt,parsep=0pt,itemsep=0pt]
        \item Lo Staff visualizza l’elenco dei mezzi filtrati.
        \end{enumerate} \\ \hline
    	\textbf{Sequenza degli eventi alternativa} & Nessuna. \\ \hline
\end{tabular}
\end{table}

\clearpage
\begin{table}[H]
\vspace*{-0cm}
\renewcommand{\arraystretch}{1.9}
\begin{tabular}{|p{3.9cm}|p{9.9cm}|}
\hline
\multicolumn{2}{|c|}{\textbf{Caso d’uso: VisualizzaMezzo}} \\ \hline
	\textbf{ID} & 16 \\ \hline
	\textbf{Breve descrizione} & Lo Staff visualizza i dati di un mezzo. \\ \hline
	\textbf{Attori primari} & Operatore \\ \hline
	\textbf{Attori secondari} & Nessuno \\ \hline
	\textbf{Precondizioni} & \begin{enumerate}[label=\arabic*.,leftmargin=14pt,labelsep=0.5em,topsep=0pt,partopsep=0pt,parsep=0pt,itemsep=0pt]
        \item L’Operatore è stato autenticato dal sistema. Il mezzo da visualizzare esiste nel sistema.
    \end{enumerate} \\ \hline
	\textbf{Sequenza degli eventi principale} & \begin{enumerate}[label=\arabic*.,leftmargin=14pt,labelsep=0.5em,topsep=0pt,partopsep=0pt,parsep=0pt,itemsep=0pt]
        \item \texttt{Include} \textit{(TrovaMezzo)}. 
        \item L’Operatore seleziona un mezzo dall’elenco. Il sistema mostra i dati del mezzo (targa, modello, stato, ecc.).
    \end{enumerate}\\ \hline
	\textbf{Postcondizioni} & \begin{enumerate}[label=\arabic*.,leftmargin=14pt,labelsep=0.5em,topsep=0pt,partopsep=0pt,parsep=0pt,itemsep=0pt]
        \item I dati del mezzo selezionato vengono visualizzati.
    \end{enumerate}\\ \hline
	\textbf{Sequenza degli eventi alternativa} & Nessuna \\ \hline
\end{tabular}
\end{table}

\clearpage
\begin{table}[H]
\vspace*{-0cm}
\renewcommand{\arraystretch}{1.9}
\begin{tabular}{|p{3.9cm}|p{9.9cm}|}
\hline
\multicolumn{2}{|c|}{\textbf{Caso d’uso: ModificaMezzo}} \\ \hline
	\textbf{ID} & 17 \\ \hline
	\textbf{Breve descrizione} & Lo Staff modifica i dati di un mezzo esistente. \\ \hline
	\textbf{Attori primari} & Staff \\ \hline
	\textbf{Attori secondari} & Nessuno \\ \hline
	\textbf{Precondizioni} & \begin{enumerate}[label=\arabic*.,leftmargin=14pt,labelsep=0.5em,topsep=0pt,partopsep=0pt,parsep=0pt,itemsep=0pt]
        \item Lo Staff è stato autenticato dal sistema. Il mezzo da modificare esiste nel sistema.
    \end{enumerate} \\ \hline
	\textbf{Sequenza degli eventi principale} & 
\begin{enumerate}[leftmargin=14pt,label=\arabic*.,labelsep=0.5em,topsep=0pt,partopsep=0pt,parsep=0pt,itemsep=0pt]
    \item \texttt{Include} \textit{(VisualizzaMezzo)}.
    \item Lo Staff modifica i dati del mezzo.
    \item \texttt{If} i dati inseriti sono validi
    \begin{enumerate}[label=\arabic{enumi}.\arabic*.,leftmargin=22pt,labelsep=0.5em,topsep=0pt,partopsep=0pt,parsep=0pt,itemsep=0pt]
        \item Il sistema aggiorna i dati del mezzo.
    \end{enumerate}
    \item \texttt{Else}
    \begin{enumerate}[label=\arabic{enumi}.\arabic*.,leftmargin=22pt,labelsep=0.5em,topsep=0pt,partopsep=0pt,parsep=0pt,itemsep=0pt]
        \item Il sistema mostra un messaggio di errore e richiede una nuova modifica.
    \end{enumerate}
\end{enumerate}\\ \hline
    	\textbf{Postcondizioni} & \begin{enumerate}[label=\arabic*.,leftmargin=14pt,labelsep=0.5em,topsep=0pt,partopsep=0pt,parsep=0pt,itemsep=0pt]
        \item I dati del mezzo risultano aggiornati nel sistema.
        \end{enumerate} \\ \hline
    	\textbf{Sequenza degli eventi alternativa} &
\begin{enumerate}[leftmargin=14pt,label=\arabic*.,labelsep=0.5em,topsep=0pt,partopsep=0pt,parsep=0pt,itemsep=0pt]
    \item \texttt{If} lo Staff non conferma l’operazione
    \begin{enumerate}[label=\arabic{enumi}.\arabic*.,leftmargin=22pt,labelsep=0.5em,topsep=0pt,partopsep=0pt,parsep=0pt,itemsep=0pt]
        \item Il sistema annulla l’operazione e non apporta modifiche.
    \end{enumerate}
\end{enumerate} \\ \hline
\end{tabular}
\end{table}

\clearpage
\begin{table}[H]
\vspace*{-0cm}
\renewcommand{\arraystretch}{1.9}
\begin{tabular}{|p{3.9cm}|p{9.9cm}|}
\hline
\multicolumn{2}{|c|}{\textbf{Caso d’uso: EliminaMezzo}} \\ \hline
	\textbf{ID} & 18 \\ \hline
	\textbf{Breve descrizione} & Lo Staff elimina un mezzo dal sistema. \\ \hline
	\textbf{Attori primari} & Staff \\ \hline
	\textbf{Attori secondari} & Nessuno \\ \hline
	\textbf{Precondizioni} & \begin{enumerate}[label=\arabic*.,leftmargin=14pt,labelsep=0.5em,topsep=0pt,partopsep=0pt,parsep=0pt,itemsep=0pt]
        \item Lo Staff è stato autenticato dal sistema. Il mezzo da eliminare esiste nel sistema.
    \end{enumerate} \\ \hline
	\textbf{Sequenza degli eventi principale} & 
\begin{enumerate}[leftmargin=14pt,label=\arabic*.,labelsep=0.5em,topsep=0pt,partopsep=0pt,parsep=0pt,itemsep=0pt]
    \item \texttt{Include} \textit{(VisualizzaMezzo)}.
    \item Lo Staff seleziona l’opzione di eliminazione.
    \item \texttt{If} lo Staff conferma l’eliminazione
    \begin{enumerate}[label=\arabic{enumi}.\arabic*.,leftmargin=22pt,labelsep=0.5em,topsep=0pt,partopsep=0pt,parsep=0pt,itemsep=0pt]
        \item Il sistema elimina il mezzo e mostra un messaggio di conferma.
    \end{enumerate}
    \item \texttt{Else}
    \begin{enumerate}[label=\arabic{enumi}.\arabic*.,leftmargin=22pt,labelsep=0.5em,topsep=0pt,partopsep=0pt,parsep=0pt,itemsep=0pt]
        \item Il sistema annulla l’operazione senza modifiche.
    \end{enumerate}
\end{enumerate}\\ \hline
    	\textbf{Postcondizioni} & \begin{enumerate}[label=\arabic*.,leftmargin=14pt,labelsep=0.5em,topsep=0pt,partopsep=0pt,parsep=0pt,itemsep=0pt]
        \item Il mezzo selezionato viene rimosso dal sistema.
        \end{enumerate} \\ \hline
    	\textbf{Sequenza degli eventi alternativa} & Nessuna. \\ \hline
\end{tabular}
\end{table}

\clearpage
\newsubsection*{Gestione Attività}

Il diagramma evidenzia le relazioni tra attori e casi d’uso per la gestione delle attività (carico/scarico rifiuti), compresa l'assegnazione di operatori e mezzi.

\begin{figure*}[ht]
    \centering
    \includegraphics[width=1\textwidth]{GestioneAttivita}
\end{figure*}

% Tabelle dei casi d'uso per Gestione Attività
\clearpage
\begin{table}[H]
\vspace*{-0cm}
\renewcommand{\arraystretch}{1.9}
\begin{tabular}{|p{3.9cm}|p{9.9cm}|}
\hline
\multicolumn{2}{|c|}{\textbf{Caso d’uso: CreaAttivita}} \\ \hline
	\textbf{ID} & 19 \\ \hline
	\textbf{Breve descrizione} & Lo Staff crea una nuova attività di gestione dei rifiuti. \\ \hline
	\textbf{Attori primari} & Staff \\ \hline
	\textbf{Attori secondari} & Nessuno \\ \hline
	\textbf{Precondizioni} & \begin{enumerate}[leftmargin=14pt,label=\arabic*.,labelsep=0.5em,topsep=0pt,partopsep=0pt,parsep=0pt,itemsep=0pt]
    \item Lo Staff è autenticato.
\end{enumerate} \\ \hline
	\textbf{Sequenza degli eventi principale} & \begin{enumerate}[leftmargin=14pt,label=\arabic*.,labelsep=0.5em,topsep=0pt,partopsep=0pt,parsep=0pt,itemsep=0pt]
    \item Il caso d’uso inizia quando Staff seleziona “Nuova Attività”.
    \item Il sistema richiede i dati dell’attività.
    \item Lo Staff inserisce i dati.
    \item \texttt{Extend} \textit{AssegnaOperatore}
    \item \texttt{Extend} \textit{AssegnaMezzo}
    \item \texttt{If} i dati sono validi
    \begin{enumerate}[label=\arabic{enumi}.\arabic*.,leftmargin=22pt,labelsep=0.5em,topsep=0pt,partopsep=0pt,parsep=0pt,itemsep=0pt]
        \item Il sistema registra la nuova attività.
        \item Il sistema mostra un messaggio di conferma.
    \end{enumerate}
    \item \texttt{Else}
    \begin{enumerate}[label=\arabic{enumi}.\arabic*.,leftmargin=22pt,labelsep=0.5em,topsep=0pt,partopsep=0pt,parsep=0pt,itemsep=0pt]
        \item Il sistema mostra un errore e richiede la correzione.
    \end{enumerate}
\end{enumerate} \\ \hline
	\textbf{Postcondizioni} & \begin{enumerate}[label=\arabic*.,leftmargin=14pt,labelsep=0.5em,topsep=0pt,partopsep=0pt,parsep=0pt,itemsep=0pt]
        \item L’attività è creata nel sistema.
    \end{enumerate} \\ \hline
	\textbf{Sequenza alternativa} & \begin{enumerate}[leftmargin=14pt,label=\arabic*.,labelsep=0.5em,topsep=0pt,partopsep=0pt,parsep=0pt,itemsep=0pt]
    \item \texttt{If} lo Staff non conferma l’operazione
    \begin{enumerate}[label=\arabic{enumi}.\arabic*.,leftmargin=22pt,labelsep=0.5em,topsep=0pt,partopsep=0pt,parsep=0pt,itemsep=0pt]
        \item Il sistema annulla l’operazione e non apporta modifiche.
    \end{enumerate}
\end{enumerate} \\ \hline
\end{tabular}
\end{table}

\clearpage
\begin{table}[H]
\vspace*{-0cm}
\renewcommand{\arraystretch}{1.9}
\begin{tabular}{|p{3.9cm}|p{9.9cm}|}
\hline
\multicolumn{2}{|c|}{\textbf{Caso d’uso: AssegnaOperatore}} \\ \hline
	\textbf{ID} & 20 \\ \hline
	\textbf{Breve descrizione} & Lo Staff assegna un Operatore ad un’attività. \\ \hline
	\textbf{Attori primari} & Staff \\ \hline
	\textbf{Attori secondari} & Nessuno \\ \hline
	\textbf{Precondizioni} & \begin{enumerate}[leftmargin=14pt,label=\arabic*.,labelsep=0.5em,topsep=0pt,partopsep=0pt,parsep=0pt,itemsep=0pt]
    \item L’attività esiste.
    \item L’operatore è registrato.
\end{enumerate} \\ \hline
	\textbf{Sequenza degli eventi principale} & \begin{enumerate}[leftmargin=14pt,label=\arabic*.,labelsep=0.5em,topsep=0pt,partopsep=0pt,parsep=0pt,itemsep=0pt]
    \item \texttt{Include} \textit{(TrovaUtente)}.
    \item Lo Staff seleziona un operatore.
    \item Lo Staff assegna l’Operatore all’attività.
    \item \texttt{If} L’Operatore è disponibile
    \begin{enumerate}[label=\arabic{enumi}.\arabic*.,leftmargin=22pt,labelsep=0.5em,topsep=0pt,partopsep=0pt,parsep=0pt,itemsep=0pt]
        \item Il sistema registra l’assegnazione.
    \end{enumerate}
    \item \texttt{Else}
    \begin{enumerate}[label=\arabic{enumi}.\arabic*.,leftmargin=22pt,labelsep=0.5em,topsep=0pt,partopsep=0pt,parsep=0pt,itemsep=0pt]
        \item Il sistema mostra il messaggio d’errore “L’operatore non è disponibile”.
    \end{enumerate}
\end{enumerate} \\ \hline
	\textbf{Postcondizioni} & \begin{enumerate}[label=\arabic*.,leftmargin=14pt,labelsep=0.5em,topsep=0pt,partopsep=0pt,parsep=0pt,itemsep=0pt]
        \item L’operatore è assegnato all’attività.
    \end{enumerate} \\ \hline
	\textbf{Seq alternativa} & \begin{enumerate}[leftmargin=14pt,label=\arabic*.,labelsep=0.5em,topsep=0pt,partopsep=0pt,parsep=0pt,itemsep=0pt]
    \item \texttt{If} lo Staff non conferma l’operazione
    \begin{enumerate}[label=\arabic{enumi}.\arabic*.,leftmargin=22pt,labelsep=0.5em,topsep=0pt,partopsep=0pt,parsep=0pt,itemsep=0pt]
        \item Il sistema annulla l’operazione e non apporta modifiche.
    \end{enumerate}
\end{enumerate} \\ \hline
\end{tabular}
\end{table}

\clearpage
\begin{table}[H]
\vspace*{-0cm}
\renewcommand{\arraystretch}{1.9}
\begin{tabular}{|p{3.9cm}|p{9.9cm}|}
\hline
\multicolumn{2}{|c|}{\textbf{Caso d’uso: AssegnaMezzo}} \\ \hline
	\textbf{ID} & 21 \\ \hline
	\textbf{Breve descrizione} & Lo Staff assegna un mezzo ad un’attività. \\ \hline
	\textbf{Attori primari} & Staff \\ \hline
	\textbf{Attori secondari} & Nessuno \\ \hline
	\textbf{Precondizioni} & \begin{enumerate}[leftmargin=14pt,label=\arabic*.,labelsep=0.5em,topsep=0pt,partopsep=0pt,parsep=0pt,itemsep=0pt]
    \item L’attività esiste.
    \item Il mezzo è registrato.
\end{enumerate} \\ \hline
	\textbf{Sequenza degli eventi principale} & \begin{enumerate}[leftmargin=14pt,label=\arabic*.,labelsep=0.5em,topsep=0pt,partopsep=0pt,parsep=0pt,itemsep=0pt]
    \item \texttt{Include} \textit{(VisualizzaMezzo)}.
    \item Lo Staff seleziona un mezzo.
    \item Lo Staff assegna il mezzo all’attività.
    \item \texttt{If} Il mezzo è disponibile
    \begin{enumerate}[label=\arabic{enumi}.\arabic*.,leftmargin=22pt,labelsep=0.5em,topsep=0pt,partopsep=0pt,parsep=0pt,itemsep=0pt]
        \item Il sistema registra l’assegnazione.
    \end{enumerate}
    \item \texttt{Else}
    \begin{enumerate}[label=\arabic{enumi}.\arabic*.,leftmargin=22pt,labelsep=0.5em,topsep=0pt,partopsep=0pt,parsep=0pt,itemsep=0pt]
        \item Il sistema mostra il messaggio d’errore “Il mezzo non è disponibile”.
    \end{enumerate}
\end{enumerate} \\ \hline
	\textbf{Postcondizioni} & \begin{enumerate}[label=\arabic*.,leftmargin=14pt,labelsep=0.5em,topsep=0pt,partopsep=0pt,parsep=0pt,itemsep=0pt]
        \item Il mezzo è associato all’attività.
    \end{enumerate} \\ \hline
	\textbf{Seq alternativa} & \begin{enumerate}[leftmargin=14pt,label=\arabic*.,labelsep=0.5em,topsep=0pt,partopsep=0pt,parsep=0pt,itemsep=0pt]
    \item \texttt{If} lo Staff non conferma l’operazione
    \begin{enumerate}[label=\arabic{enumi}.\arabic*.,leftmargin=22pt,labelsep=0.5em,topsep=0pt,partopsep=0pt,parsep=0pt,itemsep=0pt]
        \item Il sistema annulla l’operazione e non apporta modifiche.
    \end{enumerate}
\end{enumerate} \\ \hline
\end{tabular}
\end{table}

\clearpage
\begin{table}[H]
\vspace*{-0cm}
\renewcommand{\arraystretch}{1.9}
\begin{tabular}{|p{3.9cm}|p{9.9cm}|}
\hline
\multicolumn{2}{|c|}{\textbf{Caso d’uso: TrovaAttivita}} \\ \hline
	\textbf{ID} & 22 \\ \hline
	\textbf{Breve descrizione} & Lo Staff cerca un’attività. \\ \hline
	\textbf{Attori primari} & Staff \\ \hline
	\textbf{Attori secondari} & Nessuno \\ \hline
	\textbf{Precondizioni} & \begin{enumerate}[leftmargin=14pt,label=\arabic*.,labelsep=0.5em,topsep=0pt,partopsep=0pt,parsep=0pt,itemsep=0pt]
    \item Lo Staff è autenticato.
\end{enumerate} \\ \hline
	\textbf{Sequenza degli eventi principale} & \begin{enumerate}[leftmargin=14pt,label=\arabic*.,labelsep=0.5em,topsep=0pt,partopsep=0pt,parsep=0pt,itemsep=0pt]
    \item Il caso d’uso inizia quando lo Staff inserisce i criteri di ricerca.
    \item Il sistema esegue la ricerca.
    \item \texttt{If} L’attività esiste
    \begin{enumerate}[label=\arabic{enumi}.\arabic*.,leftmargin=22pt,labelsep=0.5em,topsep=0pt,partopsep=0pt,parsep=0pt,itemsep=0pt]
        \item \texttt{For Each} attività trovata
        \begin{enumerate}[label=\arabic{enumi}.\arabic{enumii}.\arabic*.,leftmargin=22pt,labelsep=0.5em,topsep=0pt,partopsep=0pt,parsep=0pt,itemsep=0pt]
            \item Il sistema mostra i risultati.
        \end{enumerate}
    \end{enumerate}
    \item \texttt{Else}
    \begin{enumerate}[label=\arabic{enumi}.\arabic*.,leftmargin=22pt,labelsep=0.5em,topsep=0pt,partopsep=0pt,parsep=0pt,itemsep=0pt]
        \item Il sistema mostra “Nessuna attività trovata”.
    \end{enumerate}
\end{enumerate} \\ \hline
	\textbf{Postcondizioni} & Nessuna. \\ \hline
	\textbf{Seq alternativa} & Nessuna. \\ \hline
\end{tabular}
\end{table}

\clearpage
\begin{table}[H]
\vspace*{-0cm}
\renewcommand{\arraystretch}{1.9}
\begin{tabular}{|p{3.9cm}|p{9.9cm}|}
\hline
\multicolumn{2}{|c|}{\textbf{Caso d’uso: VisualizzaAttivita}} \\ \hline
	\textbf{ID} & 23 \\ \hline
	\textbf{Breve descrizione} & Lo Staff visualizza l’attività selezionata. \\ \hline
	\textbf{Attori primari} & Staff \\ \hline
	\textbf{Attori secondari} & Nessuno \\ \hline
	\textbf{Precondizioni} & \begin{enumerate}[leftmargin=14pt,label=\arabic*.,labelsep=0.5em,topsep=0pt,partopsep=0pt,parsep=0pt,itemsep=0pt]
    \item Lo Staff è autenticato.
\end{enumerate} \\ \hline
	\textbf{Sequenza degli eventi principale} & \begin{enumerate}[leftmargin=14pt,label=\arabic*.,labelsep=0.5em,topsep=0pt,partopsep=0pt,parsep=0pt,itemsep=0pt]
    \item \texttt{Include} \textit{(TrovaAttivita)}.
    \item Lo Staff seleziona un’attività.
    \item Il sistema mostra le informazioni sull’attività selezionata.
\end{enumerate} \\ \hline
	\textbf{Postcondizioni} & Nessuna. \\ \hline
	\textbf{Seq alternativa} & Nessuna. \\ \hline
\end{tabular}
\end{table}

\clearpage
\begin{table}[H]
\vspace*{-0cm}
\renewcommand{\arraystretch}{1.9}
\begin{tabular}{|p{3.9cm}|p{9.9cm}|}
\hline
\multicolumn{2}{|c|}{\textbf{Caso d’uso: ModificaAttivita}} \\ \hline
	\textbf{ID} & 24 \\ \hline
	\textbf{Breve descrizione} & Lo Staff modifica un’attività esistente. \\ \hline
	\textbf{Attori primari} & Staff \\ \hline
	\textbf{Attori secondari} & Nessuno \\ \hline
	\textbf{Precondizioni} & \begin{enumerate}[leftmargin=14pt,label=\arabic*.,labelsep=0.5em,topsep=0pt,partopsep=0pt,parsep=0pt,itemsep=0pt]
    \item Lo Staff è autenticato.
    \item L’attività esiste.
\end{enumerate} \\ \hline
	\textbf{Sequenza degli eventi principale} & \begin{enumerate}[leftmargin=14pt,label=\arabic*.,labelsep=0.5em,topsep=0pt,partopsep=0pt,parsep=0pt,itemsep=0pt]
    \item \texttt{Include} \textit{(VisualizzaAttivita)}.
    \item Lo Staff aggiorna i dati dell’attività.
    \item \texttt{If} i dati sono validi
    \begin{enumerate}[label=\arabic{enumi}.\arabic*.,leftmargin=22pt,labelsep=0.5em,topsep=0pt,partopsep=0pt,parsep=0pt,itemsep=0pt]
        \item Il sistema registra le modifiche.
    \end{enumerate}
    \item \texttt{Else}
    \begin{enumerate}[label=\arabic{enumi}.\arabic*.,leftmargin=22pt,labelsep=0.5em,topsep=0pt,partopsep=0pt,parsep=0pt,itemsep=0pt]
        \item Mostra messaggio di errore.
    \end{enumerate}
\end{enumerate} \\ \hline
	\textbf{Postcondizioni} & \begin{enumerate}[label=\arabic*.,leftmargin=14pt,labelsep=0.5em,topsep=0pt,partopsep=0pt,parsep=0pt,itemsep=0pt]
        \item L’attività è aggiornata nel sistema.
    \end{enumerate} \\ \hline
	\textbf{Sequenza alternativa} & \begin{enumerate}[leftmargin=14pt,label=\arabic*.,labelsep=0.5em,topsep=0pt,partopsep=0pt,parsep=0pt,itemsep=0pt]
    \item \texttt{If} lo Staff non conferma l’operazione
    \begin{enumerate}[label=\arabic{enumi}.\arabic*.,leftmargin=22pt,labelsep=0.5em,topsep=0pt,partopsep=0pt,parsep=0pt,itemsep=0pt]
        \item Il sistema annulla l’operazione e non apporta modifiche.
    \end{enumerate}
\end{enumerate} \\ \hline
\end{tabular}
\end{table}

\clearpage
\begin{table}[H]
\vspace*{-0cm}
\renewcommand{\arraystretch}{1.9}
\begin{tabular}{|p{3.9cm}|p{9.9cm}|}
\hline
\multicolumn{2}{|c|}{\textbf{Caso d’uso: EliminaAttivita}} \\ \hline
	\textbf{ID} & 25 \\ \hline
	\textbf{Breve descrizione} & Lo Staff elimina un’attività esistente. \\ \hline
	\textbf{Attori primari} & Staff \\ \hline
	\textbf{Attori secondari} & Nessuno \\ \hline
	\textbf{Precondizioni} & \begin{enumerate}[leftmargin=14pt,label=\arabic*.,labelsep=0.5em,topsep=0pt,partopsep=0pt,parsep=0pt,itemsep=0pt]
    \item Lo Staff è autenticato.
    \item L’attività esiste.
\end{enumerate} \\ \hline
	\textbf{Sequenza degli eventi principale} & \begin{enumerate}[leftmargin=14pt,label=\arabic*.,labelsep=0.5em,topsep=0pt,partopsep=0pt,parsep=0pt,itemsep=0pt]
    \item \texttt{Include} \textit{(VisualizzaAttivita)}.
    \item Il sistema elimina l’attività selezionata.
\end{enumerate} \\ \hline
	\textbf{Postcondizioni} & \begin{enumerate}[label=\arabic*.,leftmargin=14pt,labelsep=0.5em,topsep=0pt,partopsep=0pt,parsep=0pt,itemsep=0pt]
        \item L’attività è eliminata.
    \end{enumerate} \\ \hline
	\textbf{Sequenza alternativa} & \begin{enumerate}[leftmargin=14pt,label=\arabic*.,labelsep=0.5em,topsep=0pt,partopsep=0pt,parsep=0pt,itemsep=0pt]
    \item \texttt{If} lo Staff non conferma l’operazione
    \begin{enumerate}[label=\arabic{enumi}.\arabic*.,leftmargin=22pt,labelsep=0.5em,topsep=0pt,partopsep=0pt,parsep=0pt,itemsep=0pt]
        \item Il sistema annulla l’operazione e non apporta modifiche.
    \end{enumerate}
\end{enumerate} \\ \hline
\end{tabular}
\end{table}

\clearpage

\newsubsection*{Gestione Documento}

Il diagramma evidenzia le relazioni tra attori e casi d’uso relativi alla creazione, gestione e notifica dei documenti (es. FIR, scadenze, rinnovi).

\begin{figure*}[ht]
    \centering
    \includegraphics[width=1\textwidth]{GestioneDocumento}
\end{figure*}

\clearpage
\renewcommand{\arraystretch}{1.9}
\begin{table}[H]
\vspace*{-0cm}
\begin{tabular}{|p{3.9cm}|p{9.9cm}|}
\hline
\multicolumn{2}{|c|}{\textbf{Caso d’uso: CreaDocumento}} \\ \hline
	\textbf{ID} & 26 \\ \hline
	\textbf{Breve descrizione} & Lo Staff crea un nuovo documento nel sistema. \\ \hline
	\textbf{Attori primari} & Staff \\ \hline
	\textbf{Attori secondari} & Nessuno \\ \hline
	\textbf{Precondizioni} & Lo Staff è stato autenticato dal sistema. \\ \hline
	\textbf{Sequenza degli eventi principale} &
\begin{enumerate}[leftmargin=14pt,label=\arabic*.,labelsep=0.5em,topsep=0pt,partopsep=0pt,parsep=0pt,itemsep=0pt]
    \item Il caso d’uso inizia quando lo Staff seleziona “Nuovo Documento”.
    \item Il sistema richiede i dati del documento.
    \item Lo Staff inserisce i dati.
    \item \texttt{If} i dati sono validi e non duplicati
    \begin{enumerate}[label=\arabic{enumi}.\arabic*.,leftmargin=22pt,labelsep=0.5em,topsep=0pt,partopsep=0pt,parsep=0pt,itemsep=0pt]
        \item Il sistema registra il documento.
    \end{enumerate}
    \item \texttt{Else}
    \begin{enumerate}[label=\arabic{enumi}.\arabic*.,leftmargin=22pt,labelsep=0.5em,topsep=0pt,partopsep=0pt,parsep=0pt,itemsep=0pt]
        \item Il sistema segnala un errore e richiede la correzione.
    \end{enumerate}
\end{enumerate}\\ \hline
	\textbf{Postcondizioni} & \begin{enumerate}[label=\arabic*.,leftmargin=14pt,labelsep=0.5em,topsep=0pt,partopsep=0pt,parsep=0pt,itemsep=0pt]
        \item Il documento è registrato nel sistema.
    \end{enumerate} \\ \hline
	\textbf{Sequenza degli eventi alternativa} & 
\begin{enumerate}[leftmargin=14pt,label=\arabic*.,labelsep=0.5em,topsep=0pt,partopsep=0pt,parsep=0pt,itemsep=0pt]
    \item \texttt{If} lo Staff non conferma l’operazione
    \begin{enumerate}[label=\arabic{enumi}.\arabic*.,leftmargin=22pt,labelsep=0.5em,topsep=0pt,partopsep=0pt,parsep=0pt,itemsep=0pt]
        \item Il sistema annulla l’operazione e nessun documento viene creato.
    \end{enumerate}
\end{enumerate} \\ \hline
\end{tabular}
\end{table}

\clearpage
\begin{table}[H]
\vspace*{-0cm}
\renewcommand{\arraystretch}{1.9}
\begin{tabular}{|p{3.9cm}|p{9.9cm}|}
\hline
\multicolumn{2}{|c|}{\textbf{Caso d’uso: FiltraDocumento}} \\ \hline
	\textbf{ID} & 27 \\ \hline
	\textbf{Breve descrizione} & L’Operatore applica filtri sui documenti. \\ \hline
	\textbf{Attori primari} & Operatore  \\ \hline
	\textbf{Attori secondari} & Nessuno \\ \hline
	\textbf{Precondizioni} & L’Operatore è autenticato. \\ \hline
	\textbf{Sequenza degli eventi principale} & 
\begin{enumerate}[leftmargin=14pt,label=\arabic*.,labelsep=0.5em,topsep=0pt,partopsep=0pt,parsep=0pt,itemsep=0pt]
    \item Il caso d’uso inizia quando l’Operatore applica filtri avanzati.
    \item \texttt{If} esistono documenti che rispettano i filtri
    \begin{enumerate}[label=\arabic{enumi}.\arabic*.,leftmargin=22pt,labelsep=0.5em,topsep=0pt,partopsep=0pt,parsep=0pt,itemsep=0pt]
        \item Il sistema mostra i documenti filtrati.
    \end{enumerate}
    \item \texttt{Else}
    \begin{enumerate}[label=\arabic{enumi}.\arabic*.,leftmargin=22pt,labelsep=0.5em,topsep=0pt,partopsep=0pt,parsep=0pt,itemsep=0pt]
        \item Il sistema mostra “Nessun documento corrisponde ai criteri”.
    \end{enumerate}
\end{enumerate}\\ \hline
	\textbf{Postcondizioni} & \begin{enumerate}[label=\arabic*.,leftmargin=14pt,labelsep=0.5em,topsep=0pt,partopsep=0pt,parsep=0pt,itemsep=0pt]
        \item L’Operatore visualizza i documenti filtrati.
    \end{enumerate} \\ \hline
	\textbf{Sequenza alternativa} & Nessuna. \\ \hline
\end{tabular}
\end{table}

\clearpage
\begin{table}[H]
\vspace*{-0cm}
\renewcommand{\arraystretch}{1.9}
\begin{tabular}{|p{3.9cm}|p{9.9cm}|}
\hline
\multicolumn{2}{|c|}{\textbf{Caso d’uso: TrovaDocumento}} \\ \hline
	\textbf{ID} & 28 \\ \hline
	\textbf{Breve descrizione} & L’Operatore ricerca documenti tramite criteri. \\ \hline
	\textbf{Attori primari} & Operatore \\ \hline
	\textbf{Attori secondari} & Nessuno \\ \hline
	\textbf{Precondizioni} & L’Operatore è autenticato. \\ \hline
	\textbf{Sequenza degli eventi principale} & 
\begin{enumerate}[leftmargin=14pt,label=\arabic*.,labelsep=0.5em,topsep=0pt,partopsep=0pt,parsep=0pt,itemsep=0pt]
    \item Il caso d’uso inizia quando l’Operatore inserisce un criterio di ricerca. \newline
    \texttt{Extend} \textit{FiltraDocumento}
    \item \texttt{If} Il documento esiste
    \begin{enumerate}[label=\arabic{enumi}.\arabic*.,leftmargin=22pt,labelsep=0.5em,topsep=0pt,partopsep=0pt,parsep=0pt,itemsep=0pt]
        \item \texttt{For Each} documento trovato
        \begin{enumerate}[label=\arabic{enumi}.\arabic{enumii}.\arabic*.,leftmargin=22pt,labelsep=0.5em,topsep=0pt,partopsep=0pt,parsep=0pt,itemsep=0pt]
            \item Il sistema mostra le informazioni del documento.
        \end{enumerate}
    \end{enumerate}
    \item \texttt{Else}
    \begin{enumerate}[label=\arabic{enumi}.\arabic*.,leftmargin=22pt,labelsep=0.5em,topsep=0pt,partopsep=0pt,parsep=0pt,itemsep=0pt]
        \item Il sistema mostra “Nessun documento trovato”.
    \end{enumerate}
\end{enumerate}\\ \hline
	\textbf{Postcondizioni} & Nessuna \\ \hline
	\textbf{Sequenza alternativa} & Nessuna. \\ \hline
\end{tabular}
\end{table}

\clearpage
\begin{table}[H]
\vspace*{-0cm}
\renewcommand{\arraystretch}{1.9}
\begin{tabular}{|p{3.9cm}|p{9.9cm}|}
\hline
\multicolumn{2}{|c|}{\textbf{Caso d’uso: VisualizzaDocumento}} \\ \hline
	\textbf{ID} & 29 \\ \hline
	\textbf{Breve descrizione} & L’Operatore visualizza i dati di un documento. \\ \hline
	\textbf{Attori primari} & Operatore \\ \hline
	\textbf{Attori secondari} & Nessuno \\ \hline
	\textbf{Precondizioni} & Operatore è autenticato. Il documento esiste. \\ \hline
	\textbf{Sequenza degli eventi principale} &
    \begin{enumerate}[leftmargin=14pt,label=\arabic*.,labelsep=0.5em,topsep=0pt,partopsep=0pt,parsep=0pt,itemsep=0pt]
        \item \texttt{Include} \textit{(TrovaDocumento)}. 
        \item L'Operatore seleziona un documento. Il sistema mostra i dati del documento.
    \end{enumerate} \\ \hline
	\textbf{Postcondizioni} & Nessuna. \\ \hline
	\textbf{Sequenza alternativa} & Nessuna. \\ \hline
\end{tabular}
\end{table}

\clearpage
\begin{table}[H]
\vspace*{-0cm}
\renewcommand{\arraystretch}{1.9}
\begin{tabular}{|p{3.9cm}|p{9.9cm}|}
\hline
\multicolumn{2}{|c|}{\textbf{Caso d’uso: ModificaDocumento}} \\ \hline
	\textbf{ID} & 30 \\ \hline
	\textbf{Breve descrizione} & Lo Staff modifica i dati di un documento esistente. \\ \hline
	\textbf{Attori primari} & Staff \\ \hline
	\textbf{Attori secondari} & Nessuno \\ \hline
	\textbf{Precondizioni} & Lo Staff è autenticato. Il documento esiste. \\ \hline
	\textbf{Sequenza degli eventi principale} & 
    \begin{enumerate}[leftmargin=14pt,label=\arabic*.,labelsep=0.5em,topsep=0pt,partopsep=0pt,parsep=0pt,itemsep=0pt]
        \item \texttt{Include} \textit{(VisualizzaDocumento)}. 
        \item Lo Staff modifica i dati del documento.
    \end{enumerate}
\begin{enumerate}[leftmargin=14pt,label=\arabic*.,labelsep=0.5em,topsep=0pt,partopsep=0pt,parsep=0pt,itemsep=0pt]
    \item \texttt{If} i dati modificati sono validi
    \begin{enumerate}[label=\arabic{enumi}.\arabic*.,leftmargin=22pt,labelsep=0.5em,topsep=0pt,partopsep=0pt,parsep=0pt,itemsep=0pt]
        \item Il sistema aggiorna il documento.
    \end{enumerate}
    \item \texttt{Else}
    \begin{enumerate}[label=\arabic{enumi}.\arabic*.,leftmargin=22pt,labelsep=0.5em,topsep=0pt,partopsep=0pt,parsep=0pt,itemsep=0pt]
        \item Il sistema mostra un messaggio di errore.
    \end{enumerate}
\end{enumerate}\\ \hline
	\textbf{Postcondizioni} & 
    \begin{enumerate}[leftmargin=14pt,label=\arabic*.,labelsep=0.5em,topsep=0pt,partopsep=0pt,parsep=0pt,itemsep=0pt]
        \item I dati del documento risultano aggiornati. \end{enumerate} \\ \hline
	\textbf{Sequenza alternativa} & 
\begin{enumerate}[leftmargin=14pt,label=\arabic*.,labelsep=0.5em,topsep=0pt,partopsep=0pt,parsep=0pt,itemsep=0pt]
    \item \texttt{If} lo Staff non conferma l’operazione
    \begin{enumerate}[label=\arabic{enumi}.\arabic*.,leftmargin=22pt,labelsep=0.5em,topsep=0pt,partopsep=0pt,parsep=0pt,itemsep=0pt]
        \item Il sistema annulla l’operazione e nessun documento viene modificato.
    \end{enumerate}
\end{enumerate} \\ \hline
\end{tabular}
\end{table}

\clearpage
\begin{table}[H]
\vspace*{-0cm}
\renewcommand{\arraystretch}{1.9}
\begin{tabular}{|p{3.9cm}|p{9.9cm}|}
\hline
\multicolumn{2}{|c|}{\textbf{Caso d’uso: EliminaDocumento}} \\ \hline
	\textbf{ID} & 31 \\ \hline
	\textbf{Breve descrizione} & Lo Staff elimina un documento dal sistema. \\ \hline
	\textbf{Attori primari} & Staff \\ \hline
	\textbf{Attori secondari} & Nessuno \\ \hline
	\textbf{Precondizioni} & Lo Staff è autenticato. Il documento esiste. \\ \hline
	\textbf{Sequenza degli eventi principale} & 
\begin{enumerate}[leftmargin=14pt,label=\arabic*.,labelsep=0.5em,topsep=0pt,partopsep=0pt,parsep=0pt,itemsep=0pt]
    \item \texttt{Include} \textit{(VisualizzaDocumento)}. 
    \item Lo Staff seleziona il documento da eliminare.
    \item \texttt{If} conferma l’operazione
    \begin{enumerate}[label=\arabic{enumi}.\arabic*.,leftmargin=22pt,labelsep=0.5em,topsep=0pt,partopsep=0pt,parsep=0pt,itemsep=0pt]
        \item Il sistema elimina il documento.
    \end{enumerate}
    \item \texttt{Else}
    \begin{enumerate}[label=\arabic{enumi}.\arabic*.,leftmargin=22pt,labelsep=0.5em,topsep=0pt,partopsep=0pt,parsep=0pt,itemsep=0pt]
        \item Il sistema annulla l’operazione.
    \end{enumerate}
\end{enumerate}\\ \hline
	\textbf{Postcondizioni} & \begin{enumerate}[leftmargin=14pt,label=\arabic*.,labelsep=0.5em,topsep=0pt,partopsep=0pt,parsep=0pt,itemsep=0pt]
        \item Il documento è eliminato. 
    \end{enumerate}\\ \hline
	\textbf{Sequenza alternativa} & 
\begin{enumerate}[leftmargin=14pt,label=\arabic*.,labelsep=0.5em,topsep=0pt,partopsep=0pt,parsep=0pt,itemsep=0pt]
    \item \texttt{If} lo Staff non conferma l’operazione
    \begin{enumerate}[label=\arabic{enumi}.\arabic*.,leftmargin=22pt,labelsep=0.5em,topsep=0pt,partopsep=0pt,parsep=0pt,itemsep=0pt]
        \item Il sistema annulla l’operazione e nessun documento viene eliminato.
    \end{enumerate}
\end{enumerate} \\ \hline
\end{tabular}
\end{table}

\clearpage
\begin{table}[H]
\vspace*{-0cm}
\renewcommand{\arraystretch}{1.9}
\begin{tabular}{|p{3.9cm}|p{9.9cm}|}
\hline
\multicolumn{2}{|c|}{\textbf{Caso d’uso: NotificaRinnovoCorsoSicurezza}} \\ \hline
	\textbf{ID} & 32 \\ \hline
	\textbf{Breve descrizione} & Il sistema invia notifiche per il rinnovo del corso di sicurezza. \\ \hline
	\textbf{Attori primari} & Tempo \\ \hline
	\textbf{Attori secondari} & Nessuno \\ \hline
	\textbf{Precondizioni} & Esistono corsi sicurezza con data di scadenza registrata. \\ \hline
	\textbf{Sequenza degli eventi principale} & 
\begin{enumerate}[leftmargin=14pt,label=\arabic*.,labelsep=0.5em,topsep=0pt,partopsep=0pt,parsep=0pt,itemsep=0pt]
    \item Il caso d’uso inizia quando al raggiungimento della data predefinita l’attore Tempo innesca il controllo dei corsi in scadenza.
    \item \texttt{For Each} corso in scadenza
    \begin{enumerate}[label=\arabic{enumi}.\arabic*.,leftmargin=22pt,labelsep=0.5em,topsep=0pt,partopsep=0pt,parsep=0pt,itemsep=0pt]
        \item Il sistema invia la notifica all’Operatore o allo Staff interessato.
    \end{enumerate}
\end{enumerate}\\ \hline
	\textbf{Postcondizioni} &
    \begin{enumerate}[label=\arabic*.,leftmargin=14pt,labelsep=0.5em,topsep=0pt,partopsep=0pt,parsep=0pt,itemsep=0pt]
        \item Le notifiche di rinnovo corso sicurezza sono state inviate.
    \end{enumerate} \\ \hline
	\textbf{Sequenza alternativa} & Nessuna. \\ \hline
\end{tabular}
\end{table}

\clearpage
\begin{table}[H]
\vspace*{-0cm}
\renewcommand{\arraystretch}{1.9}
\begin{tabular}{|p{3.9cm}|p{9.9cm}|}
\hline
\multicolumn{2}{|c|}{\textbf{Caso d’uso: NotificaScadenzaDocumento}} \\ \hline
	\textbf{ID} & 33 \\ \hline
	\textbf{Breve descrizione} & Il sistema invia notifiche per la scadenza dei documenti. \\ \hline
	\textbf{Attori primari} & Tempo \\ \hline
	\textbf{Attori secondari} & Nessuno \\ \hline
	\textbf{Precondizioni} & Esistono documenti con data di scadenza registrata. \\ \hline
	\textbf{Sequenza degli eventi principale} & 
\begin{enumerate}[leftmargin=14pt,label=\arabic*.,labelsep=0.5em,topsep=0pt,partopsep=0pt,parsep=0pt,itemsep=0pt]
    \item Il caso d’uso inizia quando al raggiungimento della data predefinita l’attore Tempo innesca il controllo dei documenti in scadenza.
    \item \texttt{For Each} ogni documento in scadenza
    \begin{enumerate}[label=\arabic{enumi}.\arabic*.,leftmargin=22pt,labelsep=0.5em,topsep=0pt,partopsep=0pt,parsep=0pt,itemsep=0pt]
        \item Il sistema invia la notifica allo Staff o all’Operatore.
    \end{enumerate}
\end{enumerate}\\ \hline
	\textbf{Postcondizioni} & \begin{enumerate}[label=\arabic*.,leftmargin=14pt,labelsep=0.5em,topsep=0pt,partopsep=0pt,parsep=0pt,itemsep=0pt]
        \item Le notifiche di scadenza documento sono state inviate
    \end{enumerate} \\ \hline
	\textbf{Sequenza alternativa} & Nessuna. \\ \hline
\end{tabular}
\end{table}

\clearpage
\newsection{Matrice di Mapping}

\begin{figure*}[!ht]
    \centering
    \includegraphics[width=1\textwidth]{MatriceMapping}
\end{figure*}

\newchapter{Analisi}

\newchapter{Progettazione}

\end{document}
